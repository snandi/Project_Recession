\documentclass[11pt]{extarticle} %extarticle for fontsizes other than 10, 11 And 12
%\documentclass[11p]{article}

%%%%%%%%%%%%%%%%%%%%%%%%%%%%%%%%%%%%%%%%%%%%%%%%%%%%%%%%%%%%%%%%%%%%%%
%% Input header file 
%%%%%%%%%%%%%%%%%%%%%%%%%%%%%%%%%%%%%%%%%%%%%%%%%%%%%%%%%%%%%%%%%%%%%%
/ua/snandi/TexScripts/HeaderfileTexDocs.tex

\geometry{left=1in, right=1in, top=1in, bottom=1in}

\begin{document}
\doublespacing
%\SweaveOpts{concordance=TRUE}
%\bibliographystyle{plain}  %Choose a bibliograhpic style
\bibliographystyle{chicago}

\title{Effects of Great Recession on Income Poverty}
\author{Subharati Ghosh, Subhrangshu Nandi, Susan Murphy \\
%  Statistics PhD Student, \\
%  Research Assistant,
%  Laboratory of Molecular and Computational Genomics, \\
%  University of Wisconsin - Madison}
%\date{February 16, 2015}
\date{}
}

\maketitle

\section{Study Aim}
The aim of this study is to analyze how households with a working age adult with disability compare with households with no working age adult with disability, during the {\emph{great recession}} \footnote{Business Cycle Dating Committee, National Bureau of Economic Research (NBER)}, using ``Income Poverty'' as a measure of economic wellbeing, controlling for demographic factors such as gender, marital status, education, race and origin. 

\section{Study Methods}
\subsection{Sample}
For this analysis data from US Census Bureau's SIPP 2008 panel survey was used. {\footnote{For more information on the SIPP 2008 panel schedule, please refer to this \href{http://www.census.gov/programs-surveys/sipp/data/2008-panel.html}{US Census Bureau website}}}. Questions on whether the households had a working age adult with disability were asked in wave-6 of the survey, which ended in August, 2010. Households that participated in wave-6 were included in our sample. There were a total of 34,850 households in wave six. Survey data upto wave-15 were used in our sample. Survey results from July, 2008 through June, 2013 were included in the analysis. Households whose reference person remained the same throughout the 2008 panel were kept in the sample. The reference persons of the households were also required to be 18 years or older throughout the 2008 panel. The final sample had 33,547 households that satisfied all the inclusion criteria. 

\subsection{Measures}
Total monthly household income was divided by monthly federal poverty level (FPL) and then averaged over quarters to estimate FPL100-ratio. An FPL100-ratio lower than one in any quarter indicated the household was below 100\% Federal poverty level in that quarter. Averaging monthly values over a quarter reduced the noise in the response variable by eliminating the month-by-month variability in the income data. In the sample, the monthly income data ranged from -\$27,180 to \$108,900, the average being \$5240 and median \$3,874. The negative incomes were associated with households owning business that incurred lossed in those months. The FPL100-ratio ranged from -17.95 to 89.48, with the average being 3.817 and the median 2.924. In the sample, 7,865 out of 33,547 households (23.44\%) had at least one working age adult with disability during the observed time period, as identified in wave-6 of the survey. Data from July 2008 (2008-Q3) through May 2013 (2013-Q1) were analyzed. This period overlapped with twelve of the eighteen months {\footnote{\href{http://www.nber.org/cycles/}{NBER Recession Cycles}}} of the ``Great recession'' and its long wake. 

\subsection{Analytic Strategy}
A mixed (fixed and random) effects model was fit to analyze how households with a working age adult with disability differ from households with no working age adult with disability, during the great recession, using “Income Poverty” as a measure of economic wellbeing, controlling for demographic factors. Since this dataset is longitudinal in nature, to account for between household differences a mixed effect model was use. FPL100-ratio was used as a measure of income poverty. Let $Y$ denote the vector or responses (FPL100-ratio). Let $\Theta$ denote the vector of fixed effect factors like gender, marital status, education level, race, origin of household head, along with their interactions. Let $\beta$ denote another fixed effect of time, represented as quarters, starting from 2008-Q3 and ending in 2013-Q1. Let $b$ denote the household level random effect (random intercept). The separate estimation of $b_i$ from $\epsilon_{ij}$ ensures the treatment of the two types of variability (between household, $b_i$, and within household, $\epsilon_{ij}$) separately. Then, the mixed-effects (\cite{Fitzmaurice_2012_Applied}) model for the responses, for each household $i$ can be written as
\vspace{-0.5cm}
\begin{equation}
Y_{ij} = \beta_0 + \beta t_j + X_i\Theta + b_i + \epsilon_{ij}
\label{eq:MixedEffects1}
\end{equation}
where, $\epsilon_{ij}$ are regarded as measurement errors, $i$ takes values from $1$ to $H$, the number of households, $j$ from $1$ to $T$, the total number of quarterly observations for every household. In this model, the response from the $i^{th}$ household at time $t_j$ is assumed to differ from the population mean $\beta_0 + \beta t_j + X_i\Theta$ by a household effect $b_i$ and a within household measurement error $\epsilon_{ij}$. The within household and between household errors are assumed to be normal and independent \footnote{($b_i \sim \N(0, \sigma^2_b),\ \ \epsilon_{ij} \sim \N(0, \sigma^2), \ \ b_i \indep \epsilon_{ij}, \forall i, j$)}. The effect of ``time'' is a fixed effect and it could be considered part of the fixed effects $X$. However, the question of interest is to test a linear hypothesis about the disability-by-time interaction, to detect the effect of disability on the mean response over the period of the study. Hence, the ``time'' covariate was denoted separately. During estimation it was estimated as a fixed effect. 

The bottom, or trough, of the great recession was reached in the second quarter of 2009 (marking the technical end of the recession, defined as at least two consecutive quarters of declining GDP) \footnote{Business Cycle Dating Committee, National Bureau of Economic Research (NBER)}. According to NBER, June 2009 was the final month of the recession. We wanted to check if this was reflected in the FPL100-ratio as a downward trend in the initial quarters followed by an upward trend. A simple linear term in ``time'' would be insufficient. We added a second order term $\text{time}^2$ to test the change in direction of trend. The second order term was added after centering the original ``time'' variable, to avoid introducing multicollinearity. 

Below is the final model that was fit: 
\vspace{-0.5cm}
\begin{equation}
Y_{ij} = \beta_0 + \beta_1 t_j + \beta_2 t_j^2 + \beta_D \Ind_{D_i} + \beta_t(\Ind_{D_i}*t_j) + X_i\Theta + b_i + \epsilon_{ij}
\label{eq:MixedEffects2}
\end{equation}

where, $\Ind_{D_i} = 1$, if household $i$ has a working age adult with disability, else $\Ind_{D_i} = 0$. The hypotheses of interests are: \\
$H_0: \beta_1 = 0, \text{  vs  } H_a: \beta_1 \ne 0$ tested if FPL100-ratio changed over time \\
$H_0: \beta_2 = 0, \text{  vs  } H_a: \beta_2 \ne 0$ tested if the trend of FPL100-ratio changed direction \\
$H_0: \beta_D = 0, \text{  vs  } H_a: \beta_D \ne 0$ tested if disability had any effect on FPL100-ratio\\
$H_0: \beta_t = 0, \text{  vs  } H_a: \beta_t \ne 0$ tested if disability had any effect on the slope of FPL100-ratio during the study period\\
$H_0: \Theta = 0, \text{  vs  } H_a: \Theta \ne 0$ tested if demographic factors had any effect on FPL100-ratio. \\
In addition, interactions between demographic factors, and between disability and demograhic factors were also tested. 

\noindent
{\bf{Demographic factors}}\\
The demographic factors were considered as fixed effects. The factors included in this analysis were {\emph{gender}}, {\emph{marital status}}, {\emph{education}}, and {\emph{race/origin}} of household head. The {\emph{Race/Origin}} factor included ``non-hispanic white'', ``non-hispanic black'', ``hispanic'' and ``others''. For simplicity, ``white'' and ``black'' would indicate categories ``non-hispanic white'' and ``non-hispanic black''. Gender of household head had two categories: ``male'' and ``female''. Education of household head has three categories: ``high-school or less'', ``some college, diploma, associated degrees'' and ``bachelors or higher''. Marital status of household heads has two categories: ``married'' and ``not married''. Divorced or widowed household heads were considered in the ``not married'' category.\\

\noindent
{\bf{Computational software}}\\
All analysis were conducted using the statistical software R (\cite{R}), version 3.3.1. The mixed effects models were fit using the R-package ``lme4'' (\cite{R-lme4}) and all hypothesis tests were done using the R package ``lmerTest'' (\cite{Kuznetsova_etal_2015_R-lmerTest}). The final model was fit with some of the fixed effect factors along with their interactions after performing ``backward elimination'' on the full model. Elimination of the fixed effects were done by the principle of marginality, that is: the highest order interactions are tested first: if they are significant, the lower order effects were included in the model without testing for significance. The p-values for the fixed effects are estimated from the F statistics, with ``Satterthwaite'' approximation (\cite{Satterthwaite_1946_Biometrics}) denominator degrees of freedom. The p-values for the random effect were computed from likelihood ratio tests (\cite{Morrell_1998_Biometrics}). 

\subsection{Post-hoc tests}
Post-hoc tests were conducted between categories of all demographic factors and their interactions, by calculating differences of ``Least Squares Means'' using R package ``lmerTest'' (\cite{Kuznetsova_etal_2015_R-lmerTest}), with ``Satterthwaite'' approximation (\cite{Satterthwaite_1946_Biometrics}) of the denominator degrees of freedom. \\
\noindent
{\bf{Multiple testing correction}}\\
When conducting post-hoc tests for demographic factors and their interactions, due to multiple categories of these factors the size of the tests might be inflated. Sequentially rejective {\emph{Bonferroni procedure}} (\cite{Holm_1979_SJS}) and {\emph{Benjamini-Hochberg procedure}} (\cite{Benjamini_Hochberg_1995_JRSSB}) remain the two most popular multiple testing correction procedures. Holm's sequentially rejective Bonferroni procedure controls the family-wise type-I error rate (FWER) and is more powerful than the classical Bonferroni procedure. Benjamini-Hochberg controls the false discovery rate (FDR) which is the expected value of false discovery proportion. Controlling FWER usually proves to be too conservative. Hence, we used the Benjamini-Hochberg procedure , which is less conservative, but more powerful than Bonferroni correction. All post-hoc test p-values reported are Benjamini-Hochberg corrected.

\section{Results}
Table \ref{tab:FixedEffectsBetas} shows the coefficients of time (measured in year-quarters), and disability and the interaction between them. The model includes the demographic factors, as explained in equation \ref{eq:MixedEffects2}. ANOVA table of these demographic factors is in table \ref{tab:Anova1}. The $\beta_1 = -0.0553, p < 0.01$ in the model, indicating, FPL100-ratio decreased by $0.0553$ every year, during the study period. The coefficient of {\emph{Disability}} is $\beta_D = -0.5061, p < 0.01$  indicating households with a working age adult with disability had their FPL100-ratios $0.5061$ lower, on an average, compared to households without any working age adult with disability. Next, we observe that the coefficient of interaction between time and disability ($\gamma = 0.0150, p < 0.01$) is positive. This implies that the slope of FPL100-ratio for households with disability is $-0.0443 (-0.0553 + 0.0150)$, which is less negative than the households without disability. This apparently contradictory finding leads us to conclude that households with disability although had ``significantly'' worse FPL100-ratio throughout the study period, the households without disability experienced more severe declines in their FPL100-ratios. This could throw some light on the impact of different supplementary coverage programs on households with disability. The coefficient of the quadratic term of Time ($\beta_2 = 0.0073, p < 0.01$) indicates rate of change of slope is positive. In other words, although the FPL100-ratio decreased over time (as $\beta_1 < 0$), it flattened out and even started increasing towards the latter parts of the study period. This is illustrated in figure \ref{fig:disability}. It shows the average fitted values of FPL100-ratios of households with and without a working age adult with disability. The FPL100-ratios decline sharply between 2008 and 2010, flatten out and then increase gradually after 2011. The quadratic term of Time in the model captures this behavior. It is noticable that the decline in FPL100-ratios was sharper than the gradual incline that followed. A similar behavior is observed in both types of households. 

Figures \ref{fig:gender}, \ref{fig:MS}, \ref{fig:race} and \ref{fig:education} illustrate the FPL100-ratios of the different households differentiated by gender, marital status, race and ethnicity, and education levels of the household heads. Figure \ref{fig:race} illustrates that households with ``hispanic'' household heads had minimum FPL100-ratios throughout the study period. Another important observation is the different shapes of the FPL100-ratios of the four races. Households with ``white'' heads had a gradual and steady incline in their average FPL100-ratios after 2011. However, this behavior is not observed in households with ``black'', ``hispanic'' or ``others'' heads. In figure \ref{fig:education}, households where the education level of their heads are ``high school or less'' experienced a decline in their FPL100-ratios, just like the other groups, but never experienced any improvement in the latter parts. Figure \ref{fig:contrasts} illustrates the evolutions of FPL100-ratios of two contrasting household types: one with white, married, male (with education bachelors or higher) as household heads, the other with not married, black, female (with high school or less education). 

Below are results from the post-hoc tests of all factors and their interactions.
\noindent
{\bf{Gender of household head}} \\
Gender is statistically significant. In model 1, households with a ``female'' head has, on an average, 0.113 lower FPL100-ratio ($\theta_{\text{female} - \text{male}} = -0.113, p < 0.01$) in the sample, over the study preiod. This is also illustrated in figure \ref{fig:gender}. \\
\noindent
{\bf{Marital status of household head}} \\
Marital status is statistically significant. In model 1, households with a ``not married'' head, has, on an average 0.168 lower FPL-100 ratio ($\theta_{\text{not married} - \text{married}} = -0.168, p < 0.01$ in the sample, over the study period. This is also illustrated in figure \ref{fig:MS}.\\
\noindent
{\bf{Race and Ethnicity of household head}} \\
In table \ref{tab:race_origin} we can see that regardless of disability, in Model 1, households with a ``black'' ($\theta_{\text{black} - \text{others}} = -0.54, p = 0.0000$) or ``hispanic'' ($\theta_{\text{hispanic} - \text{others}} = -0.54, p = 0.0000$) race/ethnicity as household head is worse off compared to those with ``others'' race/ethnicity. The discrepancy is higher between ``black'' and ``white'' household heads ($\theta_{\text{black} - \text{white}} = \theta_{\text{hispanic} - \text{white}} = -0.94, p = 0.0000$). This is also illustrated in figure \ref{fig:race}. \\
\noindent
{\bf{Education of household head}} \\
In table \ref{tab:education}, we see that regardless of disability, in Model 1, households in ``high school or less'' is worse off than ``some college, diploma'' ($\theta_{\text{HS} - \text{Col}} = -0.47, p = 0.0000$), which is in turn worse off than ``bachelors or higher'' ($\theta_{\text{Col} - \text{BS}} = -1.29, p = 0.0000$). This is also illustrated in figure \ref{fig:education}.\\
\noindent
{\bf{Interaction of Gender and Marital status}} \\
In table \ref{tab:gender_ms}, we see that in Model 1 ``female, not married'' households are worse off than ``male, married'' ($\theta_{\text{female, not married} - \text{male, married}} = -0.94, p = 0.0000$), than ``female, married'' ($\theta_{\text{female, not married} - \text{female, married}} = -0.78, p = 0.0000$) and ``male, not married'' ($\theta_{\text{female, not married} - \text{male, not married}} = -0.67, p = 0.0000$). \\
\noindent
{\bf{Interaction of Gender and Education}} \\
Interaction of gender and education is also statistically significant. Following from the results of the main effects, the largest differences in average FPL100-ratios between two subgroups of this interaction is that between ``female, with high-school of less'' and ``male, with bachelor or higher'' ($\theta_{\text{diff}} = -2.09, p = 0.0000 $). The other pairwise differences of average FPL100-ratios are listed in table \ref{tab:gender_education}\\
\noindent
{\bf{Interaction of Race/Ethnicity and Marital status}} \\
Table \ref{tab:ms_race_origin} lists the pairwise differences of the average FPL100-ratios between the levels of interacting race/ethnicity with marital status. The largest difference is between ``not married, black'' and ``married, white'' ($\theta_{\text{diff}} = -1.59, p = 0.0000 $), followed by that between ``not married, hispanic'' and ``married, white'' ($\theta_{\text{diff}} = -1.43, p = 0.0000 $) \\
\noindent
{\bf{Interaction of Marital status and Education}} \\
Interaction of marital status and education is also statistically significant. Table \ref{tab:ms_education} lists the pairwise differences of the average FPL100-ratios between the levels of interacting race/ethnicity with marital status. The largest difference is between ``not married, with high school or less'' and ``married, bachelors or higher'' ($\theta_{\text{diff}} = -2.21, p = 0.0000 $)\\


\section{Limitations}
\begin{enumerate}
\item Although a linear mixed effects regression model discovered some conventional and some interesting patterns in the relationships between response and demographic factors, along with disability, the trajectory of income poverty over the study period for some households were not linear. This modeling approach does not capture trajectory shapes of individual households. A non-parametric fitting of the income poverty trajectories could be tried as a pre-processing step before testing for differences in behavior between different groups of households. 
\end{enumerate}

\newpage
%% Figures for Income Poverty Report

\section{Figures}

\begin{figure}[H]
\caption{Income poverty profiles of households, by disability status}
\centering
\includegraphics[scale=0.85]{../Plots/PredictedFPLPlot_Disability.pdf}
\label{fig:disability}
\end{figure}

\begin{figure}[H]
\centering
\caption{Income poverty profiles of two most contrasting household types}
\includegraphics[scale=0.85]{../Plots/PredictedFPLPlot_Contrasts.pdf}
\label{fig:contrasts}
\begin{minipage}{0.65\textwidth} % choose width suitably
{\footnotesize {\textit{Note}}: The y-axes of the two subplots have different ranges.}
\end{minipage}
\end{figure}

%\begin{figure}[H]
%\caption{Average fitted values of FPL100-ratios, by Gender, for households with a working age adult with disability}
%\centering
%\includegraphics[scale=0.75]{../Plots/PredictedFPLPlot_Gender.pdf}
%\label{fig:gender}
%\end{figure}

%\begin{figure}[H]
%\caption{Average fitted values of FPL100-ratios, by Marital status, for households with a working age adult with disability}
%\centering
%\includegraphics[scale=0.75]{../Plots/PredictedFPLPlot_MS.pdf}
%\label{fig:MS}
%\end{figure}

%\begin{figure}[H]
%\caption{Average fitted values of FPL100-ratios, by Race/Ethnicity, for households with a working age adult with disability}
%\centering
%\includegraphics[scale=0.75]{../Plots/PredictedFPLPlot_Ethnicity.pdf}
%\label{fig:race}
%\end{figure}

%\begin{figure}[H]
%\caption{Average fitted values of FPL100-ratios, by Education, for households with a working age adult with disability}
%\centering
%\includegraphics[scale=0.75]{../Plots/PredictedFPLPlot_Education.pdf}
%\label{fig:education}
%\end{figure}

\begin{figure}[H]
\centering
\caption{Income poverty profiles for households with working age adults with disability (a) by gender (b) by marital status (c) by race and ethnicity (d) by education, of reference person}
\begin{subfigure}{0.49\linewidth}
\includegraphics[width=\textwidth]{../Plots/PredictedFPLPlot_Gender.pdf}
\caption{}
\label{fig:disab_gender}
\end{subfigure}
\begin{subfigure}{0.49\linewidth}
\includegraphics[width=\textwidth]{../Plots/PredictedFPLPlot_MS.pdf} 
\caption{}
\label{fig:disab_MS} 
\end{subfigure}
\newline
\begin{subfigure}{0.49\linewidth}
\includegraphics[width=\textwidth]{../Plots/PredictedFPLPlot_Ethnicity.pdf}
\caption{}
\label{fig:disab_race}
\end{subfigure}
\begin{subfigure}{0.49\linewidth}
\includegraphics[width=\textwidth]{../Plots/PredictedFPLPlot_Education.pdf}
\caption{}
\label{fig:disab_education}
\end{subfigure}
\label{fig:Disab_Demographics}
\end{figure}


%\begin{figure}[H]
%\caption{Participation rates in different safety net programs, in eligible households (below 200\% of FPL)}
%\centering
%\includegraphics[scale=0.85]{../Plots/ProgramParticipationPlots.pdf}
%\label{fig:programParticipation}
%\end{figure}


%% Tables for Income Poverty Report

\section{Tables}

\noindent
\begin{table}[H] 
\centering 
%\footnotesize
\caption{Description of the study sample in wave six of 2008 SIPP panel} 
\begin{tabular}{lrrrr}
\hline 
\hline 
& \multicolumn{2}{c}{\underline{\bf{HH with No Disability}}} & \multicolumn{2}{c}{\underline{\bf{HH with Disability}}} \\
{\bf{Demographic factors}} & {\bf{Number}} & {\bf{Percentage}} & {\bf{Number}} & {\bf{Percentage}} \\
\hline 
Total					& 26,104	& 77.81		& 7,443		& 22.19		\\
Gender:					&		&		&		& 		\\
\hspace{5pt} Male			& 12,626	& 48.37		& 3,378		& 45.38		\\
\hspace{5pt} Female			& 13,478	& 51.63		& 4,065		& 54.62		\\
Marital status:				&		&		&		& 		\\
\hspace{5pt} Married			& 13,555	& 51.93		& 3,968		& 53.31		\\
\hspace{5pt} Not Married		& 12,549	& 48.07		& 3,474		& 46.67		\\
Race and ethnicity:			&		&		&		& 		\\
\hspace{5pt} White			& 18,765 	& 71.89		& 4,965		& 66.71		\\
\hspace{5pt} Hispanic			&  2,996	& 11.48		&   908		& 12.20		\\
\hspace{5pt} Black			&  2,912	& 11.16		& 1,092		& 14.67		\\
\hspace{5pt} Others			&  1,431	&  5.48		&   479		&  6.44		\\
Education:				&		&		&		& 		\\
\hspace{5pt} High school or less	&  8,496 	& 32.55		& 2,994		& 40.23		\\
\hspace{5pt} Some college, diploma, assoc& 8,916 	& 34.16		& 2,863		& 38.47		\\
\hspace{5pt} Bachelors or higher	&  8,692 	& 33.30		& 1,586		& 21.31		\\
\hline 
\hline 
\end{tabular}
\label{tab:DescStats}
\end{table}


% latex table generated in R 3.3.1 by xtable 1.8-2 package
% Thu Mar 30 18:34:49 2017
\begin{table}[H]
%\footnotesize
\centering
\caption{Mixed effects regression ouput testing the difference in income poverty between families with and without a working age adult with disability, over the {\emph{great recession}} controlling for demographic factors. Only significant interactions (p-value $< 0.05$) have been reported.} 
\begin{threeparttable}
%\footnotesize
\begin{tabular}{lrcl}
  \hline
  {\bf{Predictor and Control variables}} & $\mathbf{\beta}$ & {\bf{Std. Error}} & {\bf{p-value}} \\ 
  \hline
  Intercept 						& 5.896 	& 0.037 & 0.0000 \\ 
  Time ($t$) 						& -0.054 	& 0.002 & 0.0000 \\ 
  Time-squared ($t^2$) 					& 0.007 	& 0.001 & 0.0000 \\ 
  Adult Disability 					& -0.725 	& 0.068 & $0.0000^{(5)}$ \\ 
  Adult Disability x Time 				& 0.015		& 0.004 & $0.9994^{(5)}$ \\ 
  Gender$^1$: (Female) 					& -0.368 	& 0.043 & 0.0000 \\ 
  Marital status$^2$: (Not married) 			& -0.611 	& 0.034 & 0.0000 \\ 
Race$^3$: & & & \\
  \hspace{5pt} Race2: (Black) 				& -1.284 	& 0.086 & 0.0000 \\ 
  \hspace{5pt} Race3: (Hispanic) 			& -1.505 	& 0.088 & 0.0000 \\ 
  \hspace{5pt}   Race4: (Others) 			& -0.355 	& 0.087 & 0.0000 \\ 
Education$^4$: & & & \\
  \hspace{5pt}   Education2: (Some college, diploma, assoc) & -1.516 	& 0.045 & 0.0000 \\ 
  \hspace{5pt}   Education3: (High School or less) 	& -2.204 	& 0.048 & 0.0000 \\ 
  Adult Disability x Gender 				& 0.195		& 0.050 & 0.0001 \\ 
Adult Disability x Education: & & & \\
  \hspace{5pt}   Adult Disability x Education2 		& 0.112		& 0.068 & 0.1002 \\ 
  \hspace{5pt}   Adult Disability x Education3 		& 0.219		& 0.072 & 0.0022 \\ 
  Gender x Marital status 				& -0.523 	& 0.032 & 0.0000 \\ 
Gender x Education: & & & \\
  \hspace{5pt}   Gender x Education2 			& 0.102		& 0.051 & 0.0448 \\ 
  \hspace{5pt}   Gender x Education3 			& 0.255		& 0.052 & 0.0000 \\ 
Marital status x Race: & & & \\
  \hspace{5pt}   Marital status x Race2 		& 0.209		& 0.051 & 0.0000 \\ 
  \hspace{5pt}   Marital status x Race3 		& 0.529		& 0.046 & 0.0000 \\ 
  \hspace{5pt}   Marital status x Race4 		& 0.182		& 0.069 & 0.0081 \\ 
Marital status x Education: & & & \\
  \hspace{5pt}   Marital status x Education2 		& 0.125		& 0.037 & 0.0008 \\ 
  \hspace{5pt}   Marital status x Education3 		& 0.234		& 0.038 & 0.0000 \\ 
Race x Education: & & & \\
  \hspace{5pt}   Race2 x Education2 			& 0.337		& 0.087 & 0.0001 \\ 
  \hspace{5pt}   Race3 x Education2 			& 0.401		& 0.092 & 0.0000 \\ 
  \hspace{5pt}   Race4 x Education2 			& -0.208 	& 0.104 & 0.0460 \\ 
  \hspace{5pt}   Race2 x Education3 			& 0.371		& 0.093 & 0.0001 \\ 
  \hspace{5pt}   Race3 x Education3 			& 0.499		& 0.092 & 0.0000 \\ 
  \hline
\end{tabular}
\begin{tablenotes}\footnotesize
\item[1] Base category of gender is ``Male''
\item[2] Base category of marital status is ``Married''
\item[3] Base category of race is ``White''
\item[4] Base category of education is ``Bachelors or higher''
\item[5] Note that these are p-values of one-sided tests based on hypotheses 1 and 2. 
\end{tablenotes}
\end{threeparttable}
\label{tab:Table2Reg}
\end{table}

% latex table generated in R 3.3.1 by xtable 1.8-2 package
% Fri Mar 31 18:15:28 2017
\begin{table}[H]
\centering
\caption{Mixed effects regression ouput testing the associations of demographic factors with income poverty, for families {\underline{with}} a working age adult with disability over the {\emph{great recession}}. Only significant interactions (p-value $< 0.05$) have been reported.} 
\begin{threeparttable}
\begin{tabular}{lrcr}
  \hline
  {\bf{Predictor and Control variables}} & $\mathbf{\beta}$ & {\bf{Std. Error}} & {\bf{p-value}} \\ 
  \hline
  Intercept 						& 5.361		& 0.076 & 0.0000 \\ 
  Time ($t$) 						& -0.039 	& 0.003 & 0.0000 \\ 
  Time-squared ($t^2$) 					& 0.006		& 0.002 & 0.0127 \\ 
  Gender$^1$: (Female) 					& -0.254 	& 0.088 & 0.0039 \\ 
  Marital status$^2$: (Not married) 			& -1.119 	& 0.070 & 0.0000 \\ 
Race$^3$: & & & \\
  \hspace{5pt} Race2: (Black) 				& -1.121 	& 0.167 & 0.0000 \\ 
  \hspace{5pt}   Race3: (Hispanic) 			& -1.295 	& 0.159 & 0.0000 \\ 
  \hspace{5pt}   Race4: (Others) 			& -0.325 	& 0.162 & 0.0442 \\ 
Education$^4$: & & & \\
  \hspace{5pt}   Education2: (Some college, diploma, assoc) & -1.668 	& 0.088 & 0.0000 \\ 
  \hspace{5pt}   Education3: (High School or less) 	& -2.155 	& 0.089 & 0.0000 \\ 
  Gender x Marital status 				& -0.371 	& 0.058 & 0.0000 \\ 
Gender x Education: & & & \\
  \hspace{5pt}   Gender x Education3 			& 0.223		& 0.100 & 0.0263 \\ 
Marital status x Race: & & & \\
  \hspace{5pt}   Marital status x Race3 		& 0.636		& 0.082 & 0.0000 \\ 
  \hspace{5pt}   Marital status x Race4 		& 0.288		& 0.115 & 0.0124 \\ 
Marital status x Education: & & & \\
  \hspace{5pt}   Marital status x Education2 		& 0.641		& 0.074 & 0.0000 \\ 
  \hspace{5pt}   Marital status x Education3 		& 0.759		& 0.074 & 0.0000 \\ 
Race x Education: & & & \\
  \hspace{5pt}   Race2 x Education2 			& 0.350		& 0.174 & 0.0438 \\ 
  \hline
\end{tabular}
\begin{tablenotes}\footnotesize
\item[1] Base category of gender is ``Male''
\item[2] Base category of marital status is ``Married''
\item[3] Base category of race is ``White''
\item[4] Base category of education is ``Bachelors or higher''
\end{tablenotes}
\end{threeparttable}
\label{tab:Table3Reg}
\end{table}






\newpage
\bibliography{bibTex_Reference}
\end{document}
