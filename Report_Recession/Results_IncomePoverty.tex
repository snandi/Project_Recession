\documentclass[11pt]{extarticle} %extarticle for fontsizes other than 10, 11 And 12
%\documentclass[11p]{article}

%%%%%%%%%%%%%%%%%%%%%%%%%%%%%%%%%%%%%%%%%%%%%%%%%%%%%%%%%%%%%%%%%%%%%%
%% Input header file 
%%%%%%%%%%%%%%%%%%%%%%%%%%%%%%%%%%%%%%%%%%%%%%%%%%%%%%%%%%%%%%%%%%%%%%
/ua/snandi/TexScripts/HeaderfileTexDocs.tex

\begin{document}
%\SweaveOpts{concordance=TRUE}
\bibliographystyle{plain}  %Choose a bibliograhpic style

\title{Effects of Great Recession on Income Poverty}
\author{Subharati Ghosh, Subhrangshu Nandi, Susan Murphy \\
%  Statistics PhD Student, \\
%  Research Assistant,
%  Laboratory of Molecular and Computational Genomics, \\
%  University of Wisconsin - Madison}
%\date{February 16, 2015}
\date{}
}

\maketitle

\section{Study Aim}
The aim of this study is to analyze how households with a working age adult with disability compare with households with no adult with disability, during the {\emph{great recession}}, using ``Income Poverty'' as a measure of economic wellbeing, when controlling for demographic factors such as gender, marital status, education, race and origin. 

\section{Sample}
For this analysis data from US Census Bureau's SIPP 2008 panel survey was used. {\footnote{For more information on the SIPP 2008 panel schedule, please refer to this \href{http://www.census.gov/programs-surveys/sipp/data/2008-panel.html}{US Census Bureau website}}}. Questions on whether the households had a working age adult with disability were asked in wave-6 of the survey, which ended in August, 2010. Households that participated in wave-6 were included in our sample. There were a total of 34,850 households in wave six. Survey data upto wave-15 were used in our sample. Survey results from July, 2008 through June, 2013 were included in the analysis. Households whose reference person remained the same throughout the 2008 panel were kept in the sample. The reference persons of the households were also required to be 18 years or older throughout the 2008 panel. The final sample had 33,547 households that satisfied all the inclusion criteria. 

\section{Methods}
Total monthly household income was divided by monthly federal poverty level (FPL) and then averaged over quarters to estimate FPL100 ratio. An FPL100 ratio lower than one in any quarter indicated the household was below 100\% Federal poverty level in that quarter. Averaging monthly values over a quarter reduced the month-by-month variability in the income data. In the sample, the monthly income data ranged from -\$27,180 to \$108,900, the average being \$5240 and median \$3,874. The negative incomes were associated with households owning business that incurred lossed in those months. The FPL100 ratio ranged from -17.95 to 89.48, with the average being 3.817 and the median 2.924. In the sample, there were 7,865 households (23.44\%) that had at least one working age adult with disability. 

The primary aim of this study was to estimate how the households with disability coped through the great recession, in terms of income poverty. We used the ratio of total monthly household income and the monthly federal poverty level to quantify income poverty. We name this income poverty ratio (IPR).  Households with IPRs lower than 1 were below 100\% Federal poverty levels. Data from June 2008 through May 2013 were analyzed. The baseline value of IPR (of June 2008) of each household was subtracted from the rest of that household's responses. This helped us analyze the isolated effect of the great recession on income poverty of different socio-economic strata. A mixed effect model was fit between IPR and disability, controlling for demographic variables like race, gender and marital status of household head. Since this is a panel survey, with longitudinal observations from each household, to account for ``between household'' variability, we included a random effect for each household. Our conclusion is that the great recession has had a detrimental impact on IPR. Households with unmarried female adults as the head fared much worse than other types of households. 


\end{document}
