\documentclass[11pt]{extarticle} %extarticle for fontsizes other than 10, 11 And 12
%\documentclass[11p]{article}

%%%%%%%%%%%%%%%%%%%%%%%%%%%%%%%%%%%%%%%%%%%%%%%%%%%%%%%%%%%%%%%%%%%%%%
%% Input header file 
%%%%%%%%%%%%%%%%%%%%%%%%%%%%%%%%%%%%%%%%%%%%%%%%%%%%%%%%%%%%%%%%%%%%%%
%%%%%%%%%%%%%%%%%%%%%%%% Packages %%%%%%%%%%%%%%%%%%%%%%%%
\usepackage{amscd}
\usepackage{amsmath}
\usepackage{amssymb}
\usepackage{amsthm}
\usepackage{amsxtra}
\usepackage{animate}
\usepackage{bbold}
%\usepackage{bigints}
\usepackage{caption}    %% For multiple line captions
\usepackage{color, colortbl}
\usepackage{dsfont}
\usepackage{enumerate}
\usepackage[mathscr]{eucal}
%\usepackage{fancyhdr}
\usepackage{float}
%\usepackage{fullpage}  %% Dont use this for beamer presentations
\usepackage{geometry}
\usepackage{graphicx}
\usepackage{hyperref}
\usepackage{indentfirst}
\usepackage{latexsym}
\usepackage{listings}
\usepackage{longtable}  %% to add pagebreaks in between table
\usepackage{lscape}
\usepackage{mathtools}
\usepackage{microtype}
\usepackage{multirow}
\usepackage{natbib}
\usepackage{pdfpages}
\usepackage{setspace}   %% Allows to set double or single space
\usepackage{tcolorbox}  %% For colored textboxes
\usepackage{verbatim}
\usepackage{wrapfig}
\usepackage{xargs}
\usepackage{xcolor}
\DeclareGraphicsExtensions{.pdf,.png,.jpg, .jpeg}
\definecolor{LightCyan}{rgb}{0.88,1,1}

\usepackage{array}
\newcolumntype{C}[1]{>{\centering\arraybackslash}p{#1}}  %% For wrapping text in table headers

%%%%%%%%%%%%%%%%%%%%%%%% Commands %%%%%%%%%%%%%%%%%%%%%%%%
\newcommand{\Sup}{\textsuperscript}
\newcommand{\Exp}{\mathds{E}}
\newcommand{\Prob}{\mathds{P}}
\newcommand{\Z}{\mathds{Z}}
\newcommand{\Ind}{\mathds{1}}
\newcommand{\A}{\mathcal{A}}
\newcommand{\F}{\mathcal{F}}
%\newcommand{\G}{\mathcal{G}}
\newcommand{\I}{\mathcal{I}}
\newcommand{\R}{\mathcal{R}}
\newcommand{\Y}{\mathcal{Y}}
\newcommand{\Real}{\mathbb{R}}
\newcommand{\be}{\begin{equation}}
\newcommand{\ee}{\end{equation}}
\newcommand{\bes}{\begin{equation*}}
\newcommand{\ees}{\end{equation*}}
\newcommand{\union}{\bigcup}
\newcommand{\intersect}{\bigcap}
\newcommand{\Ybar}{\overline{Y}}
\newcommand{\ybar}{\bar{y}}
\newcommand{\Xbar}{\overline{X}}
\newcommand{\xbar}{\bar{x}}
\newcommand{\betahat}{\hat{\beta}}
\newcommand{\Yhat}{\widehat{Y}}
\newcommand{\yhat}{\hat{y}}
\newcommand{\Xhat}{\widehat{X}}
\newcommand{\xhat}{\hat{x}}
\newcommand{\E}[1]{\operatorname{E}\left[ #1 \right]}
%\newcommand{\Var}[1]{\operatorname{Var}\left( #1 \right)}
\newcommand{\Var}{\operatorname{Var}}
\newcommand{\Cov}[2]{\operatorname{Cov}\left( #1,#2 \right)}
\newcommand{\N}[2][1=\mu, 2=\sigma^2]{\operatorname{N}\left( #1,#2 \right)}
\newcommand{\bp}[1]{\left( #1 \right)}
\newcommand{\bsb}[1]{\left[ #1 \right]}
\newcommand{\bcb}[1]{\left\{ #1 \right\}}
\newcommand*{\permcomb}[4][0mu]{{{}^{#3}\mkern#1#2_{#4}}}
\newcommand*{\perm}[1][-3mu]{\permcomb[#1]{P}}
\newcommand*{\comb}[1][-1mu]{\permcomb[#1]{C}}
\newcommand{\indep}{\rotatebox[origin=c]{90}{$\models$}}

\DeclareMathOperator*{\argmin}{arg\,min}

\usepackage[font=small]{caption}

\geometry{left=1in, right=1in, top=1in, bottom=1in}

\begin{document}
\doublespacing
%\SweaveOpts{concordance=TRUE}
%\bibliographystyle{plain}  %Choose a bibliograhpic style
\bibliographystyle{chicago}

\title{Effects of Great Recession on Income Poverty}
\author{Subharati Ghosh, Subhrangshu Nandi, Susan Murphy \\
%  Statistics PhD Student, \\
%  Research Assistant,
%  Laboratory of Molecular and Computational Genomics, \\
%  University of Wisconsin - Madison}
%\date{February 16, 2015}
\date{}
}

\maketitle
\section*{INTRODUCTION}
{\hl{Insert Introduction}}.

\subsubsection*{Study Aim}
The aim of this study is to analyze how households with a working age adult with disability compare with households with no working age adult with disability, during the {\emph{great recession}} \footnote{Business Cycle Dating Committee, National Bureau of Economic Research (NBER)}, using ``Income Poverty'' as a measure of economic wellbeing, controlling for demographic factors such as gender, marital status, education, race and origin. 

\section*{BACKGROUND}
{\hl{Insert Literature review and background}}.

\section*{DATA AND METHOD}
\subsubsection*{DATA}
For this analysis data from US Census Bureau's SIPP 2008 panel survey was used. {\hl{Insert SIPP details}}. {\footnote{For more information on the SIPP 2008 panel schedule, please refer to this \href{http://www.census.gov/programs-surveys/sipp/data/2008-panel.html}{US Census Bureau website}}}. There were three inclusion criteria for households in this study sample. First, the households that participated in wave six of the study were include in the study sample. Second, the reference persons of the households had to be same in all the waves the households participated in. {\hl{Give the reason why}}. Third, the reference persons had to be adults (18 years and older) throughout the household's participation in the study. {\hl{Full data from Jul 2008 through Jun 2013}}. This period overlapped with twelve of the eighteen months {\footnote{\href{http://www.nber.org/cycles/}{NBER Recession Cycles}}} of the ``Great recession'' and its long wake. In wave six, there were a total of 33,547 households that met the inclusion criteria. Of them, 7,443  households (22.16\%) had at least one working age adult with disability.

\subsubsection*{MEASURES}
\noindent
{\emph{Dependent variable}}\\
The ratio of average quarterly household income and federal poverty level (100\% FPL) was chosen as a measure of income poverty and was used as the dependent variable in our analysis. We named the dependent variable FPL100-ratio. An FPL100-ratio lower than one in any quarter indicated the household was below 100\% Federal poverty level in that quarter. In the sample, the quarterly income data ranged from -\$27,180 to \$108,900, the average being \$5240 and median \$3,874. The negative incomes were associated with households owning business that incurred lossed in those quarters. The FPL100-ratio ranged from -17.95 to 89.48, with the average being 3.817 and the median 2.924. 

\noindent
{\emph{Key Predictors}}\\
There were two key predictors in our analysis: {\emph{time}} and {\emph{adult disability}}. We analyzed how households with a working age adult with disability differed from households with no working age adult with disability, over time. Time (quarters) was treated was a continuous variable, adult disability was treated as a dichotomous factor and the interaction between time and adult disability was treated as a continuous variable. 

\noindent
{\emph{Control variables}}\\
The demographic factors like {\emph{gender}}, {\emph{marital status}}, {\emph{education}}, {\emph{race}}, and {\emph{ethnicity}} of the reference persons of the households were used as control variables our analysis. The {\emph{race/ethnicity}} factor had four categories ``non-hispanic white'', ``non-hispanic black'', ``hispanic'' and ``others''. For simplicity, ``white'' and ``black'' indicated categories ``non-hispanic white'' and ``non-hispanic black''. {\hl{Explain combining race and ethnicity}}. {\emph{Gender}} of reference persons had two categories: ``male'' and ``female''. {\emph{Education}} of reference persons had three categories: ``high-school or less'', ``some college, diploma, associated degrees'' and ``bachelors or higher''. {\emph{Marital status}} of reference persons had two categories: ``married'' and ``not married''. Divorced or widowed reference persons were considered in the ``not married'' category.

\subsubsection*{ANALYTIC STRATEGY}
A mixed (fixed and random) effects model was fit to analyze how households with a working age adult with disability differ from households with no working age adult with disability, during the great recession, using “Income Poverty” as a measure of economic wellbeing, controlling for demographic factors. Since this dataset is longitudinal in nature, to account for ``between household'' differences a mixed effect model was used. $Y$ denoted the vector of responses (FPL100-ratio). $\Theta$ denoted the vector of fixed effect factors like gender, marital status, education level, race/ethnicity of reference person, along with their interactions. $\beta$ denoted another fixed effect of time (in quarters), starting from 2008-Q3 and ending in 2013-Q1. $b$ denoted the household level random effect (random intercept). The separate estimation of $b_i$ from $\epsilon_{ij}$ ensures the separate estimation of the two types of variability (between household, $b_i$, and within household, $\epsilon_{ij}$). For example, a simple mixed-effects model for the analysis could be written as
\vspace{-0.5cm}
\begin{equation}
Y_{ij} = \beta_0 + \beta t_j + X_i\Theta + b_i + \epsilon_{ij}
\label{eq:MixedEffects1}
\end{equation}
where, $\epsilon_{ij}$ are measurement errors, $i$ ranges from $1$ to $H$, the number of households, $j$ from $1$ to $T$, the total number of quarters. In this model, the response from the $i^{th}$ household at time $t_j$ is assumed to differ from the population mean $\beta_0 + \beta t_j + X_i\Theta$ by a household effect $b_i$ and a within household measurement error $\epsilon_{ij}$. The within household and between household errors are assumed to be normal and independent \footnote{($b_i \sim \N(0, \sigma^2_b),\ \ \epsilon_{ij} \sim \N(0, \sigma^2), \ \ b_i \indep \epsilon_{ij}, \forall i, j$)}. ``time'' was a fixed effect. 

The  trough, of the great recession was reached in the second quarter of 2009 (marking the technical end of the recession, defined as at least two consecutive quarters of declining GDP) \footnote{Business Cycle Dating Committee, National Bureau of Economic Research (NBER)}. According to NBER, June 2009 was the final month of the recession. We checked if this was reflected in the FPL100-ratio as a downward trend in the initial quarters followed by an upward trend. A linear term in ``time'' was insufficient to capture this effect. We added a second order term $\text{time}^2$ to test the change in direction of trend. The second order term was added after centering the original ``time'' variable, to avoid introducing multicollinearity. An indicator variable was used to denote the presence of working age adult with disability in a household. An interaction term between this indicator variable and time was also included to estimate the difference in slopes between households with and without a working age adult with disability. Below is the final model that was fit: 
\vspace{-0.5cm}
\begin{equation}
Y_{ij} = \beta_0 + \beta_1 t_j + \beta_2 t_j^2 + \beta_D \Ind_{D_i} + \beta_t(\Ind_{D_i}*t_j) + X_i\Theta + b_i + \epsilon_{ij}
\label{eq:MixedEffects2}
\end{equation}

where, $\Ind_{D_i} = 1$, if household $i$ has a working age adult with disability, else $\Ind_{D_i} = 0$. The hypotheses of interests are:
\begin{enumerate}
\item $H_0: \beta_1 = 0, \text{  vs  } H_a: \beta_1 \ne 0$ tested if FPL100-ratio changed over time
\item $H_0: \beta_2 = 0, \text{  vs  } H_a: \beta_2 \ne 0$ tested if the trend of FPL100-ratio changed direction
\item $H_0: \beta_D = 0, \text{  vs  } H_a: \beta_D \ne 0$ tested if disability had any effect on FPL100-ratio
\item $H_0: \beta_t = 0, \text{  vs  } H_a: \beta_t \ne 0$ tested if disability had any effect on the slope of FPL100-ratio during the study period
\item $H_0: \Theta = 0, \text{  vs  } H_a: \Theta \ne 0$ tested if demographic factors had any effect on FPL100-ratio.
\end{enumerate}
In addition, interactions between demographic factors, and between disability and demograhic factors were also tested. The demographic factors were considered as fixed effects.

The final model was fit with some of the fixed effect factors along with their interactions after performing ``backward elimination'' on the full model. Elimination of the fixed effects were done by the principle of marginality, that is: the highest order interactions are tested first: if they are significant, the lower order effects were included in the model without testing for significance. The p-values for the fixed effects are estimated from the F statistics, with ``Satterthwaite'' approximation (\cite{Satterthwaite_1946_Biometrics}) denominator degrees of freedom. The p-values for the random effect were computed from likelihood ratio tests (\cite{Morrell_1998_Biometrics}). 

\noindent
{\bf{Post-hoc tests}} \\
Post-hoc tests were conducted between categories of all demographic factors and their interactions, by calculating differences of ``Least Squares Means'' using R package ``lmerTest'' (\cite{Kuznetsova_etal_2015_R-lmerTest}), with ``Satterthwaite'' approximation (\cite{Satterthwaite_1946_Biometrics}) of the denominator degrees of freedom.

\noindent
{\bf{Multiple testing correction}}\\
When conducting post-hoc tests for demographic factors and their interactions, due to multiple categories of these factors the size of the tests could be inflated. Sequentially rejective {\emph{Bonferroni procedure}} (\cite{Holm_1979_SJS}) and {\emph{Benjamini-Hochberg procedure}} (\cite{Benjamini_Hochberg_1995_JRSSB}) remain the two most popular multiple testing correction procedures. Holm's sequentially rejective Bonferroni procedure controls the family-wise type-I error rate (FWER) and is more powerful than the classical Bonferroni procedure. Benjamini-Hochberg controls the false discovery rate (FDR) which is the expected value of false discovery proportion. Controlling FWER usually proves to be too conservative. Hence, we used the Benjamini-Hochberg procedure, which is less conservative, but more powerful than Bonferroni correction. All post-hoc test p-values reported were Benjamini-Hochberg corrected.

\noindent
{\bf{Computational software}}\\
All analysis were conducted using the statistical software R (\cite{R}), version 3.3.1. The mixed effects models were fit using the R-package ``lme4'' (\cite{R-lme4}) and all hypothesis tests were done using the R package ``lmerTest'' (\cite{Kuznetsova_etal_2015_R-lmerTest}). 

\section*{RESULTS}
Table \ref{tab:FixedEffectsBetas} shows the coefficients of time (measured in year-quarters), and disability and the interaction between them. The model includes the demographic factors, as explained in equation \ref{eq:MixedEffects2}. ANOVA table of these demographic factors is in table \ref{tab:Anova1}. The $\beta_1 = -0.0553, p < 0.01$ in the model, indicating, FPL100-ratio decreased by $0.0553$ every year, during the study period. The coefficient of {\emph{Disability}} is $\beta_D = -0.5061, p < 0.01$  indicating households with a working age adult with disability had their FPL100-ratios $0.5061$ lower, on an average, compared to households without any working age adult with disability. Next, we observe that the coefficient of interaction between time and disability ($\gamma = 0.0150, p < 0.01$) is positive. This implies that the slope of FPL100-ratio for households with disability is $-0.0443 (-0.0553 + 0.0150)$, which is less negative than the households without disability. This apparently contradictory finding leads us to conclude that households with disability although had ``significantly'' worse FPL100-ratio throughout the study period, the households without disability experienced more severe declines in their FPL100-ratios. This could throw some light on the impact of different supplementary coverage programs on households with disability. The coefficient of the quadratic term of Time ($\beta_2 = 0.0073, p < 0.01$) indicates rate of change of slope is positive. In other words, although the FPL100-ratio decreased over time (as $\beta_1 < 0$), it flattened out and even started increasing towards the latter parts of the study period. This is illustrated in figure \ref{fig:disability}. It shows the average fitted values of FPL100-ratios of households with and without a working age adult with disability. The FPL100-ratios decline sharply between 2008 and 2010, flatten out and then increase gradually after 2011. The quadratic term of Time in the model captures this behavior. It is noticable that the decline in FPL100-ratios was sharper than the gradual incline that followed. A similar behavior is observed in both types of households. 

Figures \ref{fig:disab_gender}, \ref{fig:disab_MS}, \ref{fig:disab_race} and \ref{fig:disab_education} illustrate the FPL100-ratios of the different households differentiated by gender, marital status, race and ethnicity, and education levels of the reference persons. Figure \ref{fig:disab_race} illustrates that households with ``hispanic'' reference persons had minimum FPL100-ratios throughout the study period. Another important observation is the different shapes of the FPL100-ratios of the four races. Households with ``white'' heads had a gradual and steady incline in their average FPL100-ratios after 2011. However, this behavior is not observed in households with ``black'', ``hispanic'' or ``others'' heads. In figure \ref{fig:disab_education}, households where the education level of their heads are ``high school or less'' experienced a decline in their FPL100-ratios, just like the other groups, but never experienced any improvement in the latter parts. Figure \ref{fig:contrasts} illustrates the evolutions of FPL100-ratios of two contrasting household types: one with white, married, male (with education bachelors or higher) as reference persons, the other with not married, black, female (with high school or less education). 

Below are results from the post-hoc tests of all factors and their interactions. \\
\noindent
{\bf{Gender of reference person}} \\
Gender is statistically significant. In model 1, households with a ``female'' head has, on an average, 0.113 lower FPL100-ratio ($\theta_{\text{female} - \text{male}} = -0.113, p < 0.01$) in the sample, over the study preiod. This is also illustrated in figure \ref{fig:disab_gender}. \\
\noindent
{\bf{Marital status of reference person}} \\
Marital status is statistically significant. In model 1, households with a ``not married'' head, has, on an average 0.168 lower FPL-100 ratio ($\theta_{\text{not married} - \text{married}} = -0.168, p < 0.01$ in the sample, over the study period. This is also illustrated in figure \ref{fig:disab_MS}.\\
\noindent
{\bf{Race and Ethnicity of reference person}} \\
In table \ref{tab:race_origin} we can see that regardless of disability, in Model 1, households with a ``black'' ($\theta_{\text{black} - \text{others}} = -0.54, p = 0.0000$) or ``hispanic'' ($\theta_{\text{hispanic} - \text{others}} = -0.54, p = 0.0000$) race/ethnicity as reference person is worse off compared to those with ``others'' race/ethnicity. The discrepancy is higher between ``black'' and ``white'' reference persons ($\theta_{\text{black} - \text{white}} = \theta_{\text{hispanic} - \text{white}} = -0.94, p = 0.0000$). This is also illustrated in figure \ref{fig:disab_race}. \\
\noindent
{\bf{Education of reference person}} \\
In table \ref{tab:education}, we see that regardless of disability, in Model 1, households in ``high school or less'' is worse off than ``some college, diploma'' ($\theta_{\text{HS} - \text{Col}} = -0.47, p = 0.0000$), which is in turn worse off than ``bachelors or higher'' ($\theta_{\text{Col} - \text{BS}} = -1.29, p = 0.0000$). This is also illustrated in figure \ref{fig:disab_education}.\\
\noindent
{\bf{Interaction of Gender and Marital status}} \\
In table \ref{tab:gender_ms}, we see that in Model 1 ``female, not married'' households are worse off than ``male, married'' ($\theta_{\text{female, not married} - \text{male, married}} = -0.94, p = 0.0000$), than ``female, married'' ($\theta_{\text{female, not married} - \text{female, married}} = -0.78, p = 0.0000$) and ``male, not married'' ($\theta_{\text{female, not married} - \text{male, not married}} = -0.67, p = 0.0000$). \\
\noindent
{\bf{Interaction of Gender and Education}} \\
Interaction of gender and education is also statistically significant. Following from the results of the main effects, the largest differences in average FPL100-ratios between two subgroups of this interaction is that between ``female, with high-school of less'' and ``male, with bachelor or higher'' ($\theta_{\text{diff}} = -2.09, p = 0.0000 $). The other pairwise differences of average FPL100-ratios are listed in table \ref{tab:gender_education}\\
\noindent
{\bf{Interaction of Race/Ethnicity and Marital status}} \\
Table \ref{tab:ms_race_origin} lists the pairwise differences of the average FPL100-ratios between the levels of interacting race/ethnicity with marital status. The largest difference is between ``not married, black'' and ``married, white'' ($\theta_{\text{diff}} = -1.59, p = 0.0000 $), followed by that between ``not married, hispanic'' and ``married, white'' ($\theta_{\text{diff}} = -1.43, p = 0.0000 $) \\
\noindent
{\bf{Interaction of Marital status and Education}} \\
Interaction of marital status and education is also statistically significant. Table \ref{tab:ms_education} lists the pairwise differences of the average FPL100-ratios between the levels of interacting race/ethnicity with marital status. The largest difference is between ``not married, with high school or less'' and ``married, bachelors or higher'' ($\theta_{\text{diff}} = -2.21, p = 0.0000 $)\\


\section*{Limitations}
\begin{enumerate}
\item Although a linear mixed effects regression model discovered some conventional and some interesting patterns in the relationships between response and demographic factors, along with disability, the trajectory of income poverty over the study period for some households were not linear. This modeling approach did not capture trajectory shapes of individual households. A non-parametric fitting of the income poverty trajectories could be tried as a pre-processing step before testing for differences in behavior between different groups of households. 

\item Some households in the sample did not participate over all the waves. Since households that participated in wave six were included there were some households that were first interviewed in wave six and some that were no longer interviewed after wave six. There were no means of determining the reasons for dropping out from the survey, nor the reasons for late inception into the survey. Since the {\emph{great recession}} was a significant economic and social event, we included households without complete participation in order to maximize the sample size, and incorporate the effect of the recession on more households. If, however, the reasons for dropping out or late joining had an association with the outcome of the study (income poverty), including those households could increase bias in the estimates, in spite of the estimates being more stable (less variance). Chapter 2 in SIPP users guide \footnote{https://www2.census.gov/programs-surveys/sipp/guidance/SIPP\_2008\_USERS\_Guide\_Chapter2.pdf} mentions that the survey weights are adjusted to account for some types of household nonresponse with the objective of ameliorating the nonresponse bias. 
\end{enumerate}

\newpage
%% Figures for Income Poverty Report

\section{Figures}

\begin{figure}[H]
\caption{Average fitted values of FPL100-ratios, by Disability status}
\centering
\includegraphics[scale=0.85]{../Plots/PredictedFPLPlot_Disability.pdf}
\label{fig:disability}
\end{figure}

\begin{figure}[H]
\caption{Average fitted values of FPL100-ratios, by Gender, for households with a working age adult with disability}
\centering
\includegraphics[scale=0.75]{../Plots/PredictedFPLPlot_Gender.pdf}
\label{fig:gender}
\end{figure}

\begin{figure}[H]
\caption{Average fitted values of FPL100-ratios, by Marital status, for households with a working age adult with disability}
\centering
\includegraphics[scale=0.75]{../Plots/PredictedFPLPlot_MS.pdf}
\label{fig:MS}
\end{figure}

\begin{figure}[H]
\caption{Average fitted values of FPL100-ratios, by Race/Ethnicity, for households with a working age adult with disability}
\centering
\includegraphics[scale=0.75]{../Plots/PredictedFPLPlot_Ethnicity.pdf}
\label{fig:race}
\end{figure}

\begin{figure}[H]
\caption{Average fitted values of FPL100-ratios, by Education, for households with a working age adult with disability}
\centering
\includegraphics[scale=0.75]{../Plots/PredictedFPLPlot_Education.pdf}
\label{fig:education}
\end{figure}

\begin{figure}[H]
\caption{Average fitted values of FPL100-ratios, for two contrasting household types}
\centering
\includegraphics[scale=0.85]{../Plots/PredictedFPLPlot_Contrasts.pdf}
\label{fig:contrasts}
\end{figure}


%% Tables for Income Poverty Report

\section{Tables}

\noindent
\begin{table}[H] 
\centering 
\footnotesize
%\begin{tabular}{@{\extracolsep{5pt}}lccc|c}
\begin{tabular}{l|c}
\hline 
\hline 
% & \multicolumn{4}{c}{\textit{Dependent variable:}} \\ 
%& Model1 (with Baseline) & Model2 (with No Baseline) \\
%\hline 
%Intercept		&	$2.779^{***}$		&	$0.1478^{**}$ 		\\
%			&	$(0.0706)$		&	$(0.0646)$		\\
%time (Year-Quarter)	&	$-0.0553^{***}$		&	$-0.0543^{***}$ 	\\
%			&	$(0.0021)$		&	$(0.0021)$		\\
%$\text{time}^2$		&	$0.0073^{***}$		&	$0.0075^{***}$ 		\\
%			&	$(0.0013)$		&	$(0.0013)$		\\
%Disability		&	$-0.5061^{***}$		&	$-0.0233$		\\    
%			&	$(0.0564)$		&	$(0.0499)$		\\
%Year-Quarter*Disability	&	$0.0150^{***}$		&	$0.0150^{***}$		\\
%			&	$(0.0043)$		&	$(0.0043)$		\\
%
Continuous Indep Variables & Model coefficients \\
\hline 
Intercept		&	$2.779^{***}$		\\
			&	$(0.0706)$		\\
time (Year-Quarter)	&	$-0.0553^{***}$		\\
			&	$(0.0021)$		\\
$\text{time}^2$		&	$0.0073^{***}$		\\
			&	$(0.0013)$		\\
Disability		&	$-0.5061^{***}$		\\    
			&	$(0.0564)$		\\
Year-Quarter*Disability	&	$0.0150^{***}$		\\
			&	$(0.0043)$		\\
\hline 
\hline 
\multicolumn{2}{r}{\textit{Note:}  $^{*}$p$<$0.1; $^{**}$p$<$0.05; $^{***}$p$<$0.01} \\ 

\end{tabular}
\caption{Coefficients of Fixed effects in regression models} 
\label{tab:FixedEffectsBetas} 
\end{table}


% latex table generated in R 3.3.1 by xtable 1.8-2 package
% Mon Mar 13 12:08:48 2017
\begin{table}[H]
\footnotesize
\centering
\begin{tabular}{lrrrrr}
  \hline
  & Sum Sq & Mean Sq & NumDF & F.value & p.value \\ 
  \hline
  Time & 5380.93 & 5380.93 & 1 & 446.58 & 0.0000 \\ 
  Disability & 2280.29 & 2280.29 & 1 & 189.25 & 0.0000 \\ 
  Gender & 3143.82 & 3143.82 & 1 & 260.92 & 0.0000 \\ 
  Marital status & 5788.34 & 5788.34 & 1 & 480.40 & 0.0000 \\ 
  Ethnicity & 8743.91 & 2914.64 & 3 & 241.90 & 0.0000 \\ 
  Education & 15873.19 & 7936.60 & 2 & 658.69 & 0.0000 \\ 
  Gender \& Marital status & 3184.05 & 3184.05 & 1 & 264.26 & 0.0000 \\ 
  Marital status \& Ethnicity & 1670.73 & 556.91 & 3 & 46.22 & 0.0000 \\ 
  Ethnicity \& Education & 652.73 & 108.79 & 6 & 9.03 & 0.0000 \\ 
  $\text{Time}^2$ & 414.33 & 414.33 & 1 & 34.39 & 0.0000 \\ 
  Marital status \& Education & 450.29 & 225.15 & 2 & 18.69 & 0.0000 \\ 
  Gender \& Education & 310.12 & 155.06 & 2 & 12.87 & 0.0000 \\ 
  Disability \& Gender & 182.78 & 182.78 & 1 & 15.17 & 0.0001 \\ 
  Time:adult\_disb & 143.66 & 143.66 & 1 & 11.92 & 0.0006 \\ 
  Disability \& Education & 116.32 & 58.16 & 2 & 4.83 & 0.0080 \\ 
   \hline
\end{tabular}
\caption{FPL100 vs demographic factors and time and disability status} 
\label{tab:Anova1}
\end{table}


% latex table generated in R 3.3.1 by xtable 1.8-2 package
% Mon Mar 13 12:10:56 2017
\begin{table}[H]
\footnotesize
\centering
\begin{tabular}{lrrrrr}
  \hline
  & Sum Sq & Mean Sq & NumDF & F.value & p.value \\ 
  \hline
  Marital status & 66.83 & 66.83 & 1 & 316.81 & 0.0000 \\ 
  Education & 65.21 & 32.60 & 2 & 154.56 & 0.0000 \\ 
  Marital status \& Ethnicity & 18.18 & 6.06 & 3 & 28.73 & 0.0000 \\ 
  Ethnicity & 14.34 & 4.78 & 3 & 22.66 & 0.0000 \\ 
  Gender & 9.09 & 9.09 & 1 & 43.08 & 0.0000 \\ 
  Ethnicity \& Education & 11.36 & 1.89 & 6 & 8.98 & 0.0000 \\ 
  Marital status \& Education & 8.03 & 4.01 & 2 & 19.03 & 0.0000 \\ 
  Gender \& Marital status & 4.40 & 4.40 & 1 & 20.86 & 0.0000 \\ 
  Time & 3.39 & 3.39 & 1 & 16.08 & 0.0001 \\ 
  $\text{Time}^2$ & 0.06 & 0.06 & 1 & 0.26 & 0.6074 \\ 
  \hline
\end{tabular}
\caption{FPL100 vs demographic factors and time, for households with Disability} 
\label{tab:Anova2}
\end{table}


% latex table generated in R 3.3.1 by xtable 1.8-2 package
% Mon Mar 13 12:17:15 2017
\begin{table}[H]
\footnotesize
\centering
\begin{tabular}{lrrrrr}
  \hline
  & Sum Sq & Mean Sq & NumDF & F.value & p.value \\ 
  \hline
  Time & 13.16 & 13.16 & 1 & 86.80 & 0.0000 \\ 
  Gender & 48.49 & 48.49 & 1 & 319.96 & 0.0000 \\ 
  Marital status & 76.70 & 76.70 & 1 & 506.08 & 0.0000 \\ 
  Ethnicity & 99.18 & 33.06 & 3 & 218.15 & 0.0000 \\ 
  Education & 106.34 & 53.17 & 2 & 350.84 & 0.0000 \\ 
  Gender \& Marital status & 12.85 & 12.85 & 1 & 84.80 & 0.0000 \\ 
  Marital status \& Education & 27.95 & 13.97 & 2 & 92.21 & 0.0000 \\ 
  Ethnicity \& Education & 25.03 & 4.17 & 6 & 27.52 & 0.0000 \\ 
  Marital status \& Ethnicity & 9.60 & 3.20 & 3 & 21.12 & 0.0000 \\ 
  I(Time\verb|^|2) & 0.03 & 0.03 & 1 & 0.21 & 0.6489 \\ 
   \hline
\end{tabular}
\caption{FPL100 vs demographic factors and time, for households with No Disability} 
\label{tab:Anova3}
\end{table}

% latex table generated in R 3.3.1 by xtable 1.8-2 package
% Mon Mar 13 12:19:44 2017
\begin{table}[H]
\footnotesize
\centering
\begin{tabular}{lrrrr}
  \hline
  Factor Levels & Estimate & Standard Error & t-value & p-value \\ 
  \hline
   Black - Others & -0.54 & 0.07 & -8.02 & 0.0000 \\ 
   Black - White & -0.94 & 0.05 & -20.16 & 0.0000 \\ 
   Hispanic - Others & -0.54 & 0.07 & -7.78 & 0.0000 \\ 
   Hispanic - White & -0.94 & 0.05 & -19.60 & 0.0000 \\ 
   Others - White & -0.40 & 0.05 & -7.43 & 0.0000 \\ 
   Black - Hispanic & -0.00 & 0.06 & -0.05 & 0.9611 \\ 
  \hline
\end{tabular}
\caption{Post-hoc test of Ethnicity} 
\label{tab:race_origin}
\end{table}

% latex table generated in R 3.3.1 by xtable 1.8-2 package
% Mon Mar 13 12:21:17 2017
\begin{table}[H]
\footnotesize
\centering
\begin{tabular}{lrrrr}
  \hline
Factor Levels & Estimate & Standard Error & t-value & p-value \\ 
  \hline
 High School or less - Some college, diploma, assoc & -0.47 & 0.04 & -12.47 & 0.0000 \\ 
   High School or less - Bachelors or higher & -1.69 & 0.05 & -36.04 & 0.0000 \\ 
   Some college, diploma, assoc - Bachelors or higher & -1.21 & 0.04 & -27.03 & 0.0000 \\ 
   \hline
\end{tabular}
\caption{Post-hoc test of Education} 
\label{tab:education}
\end{table}

% latex table generated in R 3.3.1 by xtable 1.8-2 package
% Mon Mar 13 12:23:35 2017
\begin{table}[H]
\footnotesize
\centering
\begin{tabular}{lrrrr}
  \hline
Factor Levels & Estimate & Standard Error & t-value & p-value \\ 
  \hline
  Male Married -  Female Married & 0.15 & 0.03 & 5.03 & 0.0000 \\ 
    Male Married -  Male Not married & 0.26 & 0.03 & 9.00 & 0.0000 \\ 
    Male Married -  Female Not married & 0.94 & 0.03 & 27.73 & 0.0000 \\ 
    Female Married -  Female Not married & 0.78 & 0.03 & 27.52 & 0.0000 \\ 
    Male Not married -  Female Not married & 0.67 & 0.03 & 22.30 & 0.0000 \\ 
    Female Married -  Male Not married & 0.11 & 0.04 & 3.03 & 0.0024 \\ 
   \hline
\end{tabular}
\caption{Post-hoc test of Gender and Marital status} 
\label{tab:gender:ms}
\end{table}

% latex table generated in R 3.3.1 by xtable 1.8-2 package
% Mon Mar 13 12:28:05 2017
\begin{table}[H]
\footnotesize
\centering
\begin{tabular}{lrrrr}
  \hline
Factor Levels & Estimate & Standard Error & t-value & p-value \\ 
  \hline
  Male High School or less -  Female High School or less & 0.28 & 0.03 & 7.98 & 0.0000 \\ 
    Male High School or less -  Male Some college, diploma, assoc & -0.55 & 0.05 & -12.14 & 0.0000 \\ 
    Male High School or less -  Male Bachelors or higher & -1.81 & 0.05 & -33.25 & 0.0000 \\ 
    Male High School or less -  Female Bachelors or higher & -1.28 & 0.05 & -23.46 & 0.0000 \\ 
    Female High School or less -  Male Some college, diploma, assoc & -0.82 & 0.05 & -17.49 & 0.0000 \\ 
    Female High School or less -  Female Some college, diploma, assoc & -0.39 & 0.04 & -9.01 & 0.0000 \\ 
    Female High School or less -  Male Bachelors or higher & -2.09 & 0.06 & -37.35 & 0.0000 \\ 
    Female High School or less -  Female Bachelors or higher & -1.56 & 0.05 & -29.82 & 0.0000 \\ 
    Male Some college, diploma, assoc -  Female Some college, diploma, assoc & 0.43 & 0.04 & 11.59 & 0.0000 \\ 
    Male Some college, diploma, assoc -  Male Bachelors or higher & -1.26 & 0.05 & -23.80 & 0.0000 \\ 
    Male Some college, diploma, assoc -  Female Bachelors or higher & -0.73 & 0.05 & -13.44 & 0.0000 \\ 
    Female Some college, diploma, assoc -  Male Bachelors or higher & -1.69 & 0.05 & -30.91 & 0.0000 \\ 
    Female Some college, diploma, assoc -  Female Bachelors or higher & -1.16 & 0.05 & -23.24 & 0.0000 \\ 
    Male Bachelors or higher -  Female Bachelors or higher & 0.53 & 0.04 & 12.35 & 0.0000 \\ 
    Male High School or less -  Female Some college, diploma, assoc & -0.12 & 0.05 & -2.56 & 0.0105 \\ 
   \hline
\end{tabular}
\caption{Post-hoc test of Gender and Education} 
\label{tab:gender_education}
\end{table}

% latex table generated in R 3.3.1 by xtable 1.8-2 package
% Mon Mar 13 12:35:52 2017
\begin{table}[H]
\footnotesize
\centering
\begin{tabular}{lrrrr}
  \hline
Factor Levels & Estimate & Standard Error & t-value & p-value \\ 
  \hline
  Married Black -  Not married Black & 0.54 & 0.05 & 11.34 & 0.0000 \\ 
    Married Black -  Not married Hispanic & 0.38 & 0.07 & 5.12 & 0.0000 \\ 
    Married Black -  Married Others & -0.55 & 0.08 & -6.83 & 0.0000 \\ 
    Married Black -  Married White & -1.05 & 0.06 & -18.44 & 0.0000 \\ 
    Married Black -  Not married White & -0.30 & 0.06 & -5.18 & 0.0000 \\ 
    Not married Black -  Married Hispanic & -0.39 & 0.07 & -5.70 & 0.0000 \\ 
    Not married Black -  Married Others & -1.10 & 0.08 & -14.39 & 0.0000 \\ 
    Not married Black -  Not married Others & -0.53 & 0.08 & -6.88 & 0.0000 \\ 
    Not married Black -  Married White & -1.59 & 0.05 & -32.10 & 0.0000 \\ 
    Not married Black -  Not married White & -0.84 & 0.05 & -16.91 & 0.0000 \\ 
    Married Hispanic -  Not married Hispanic & 0.22 & 0.04 & 5.25 & 0.0000 \\ 
    Married Hispanic -  Married Others & -0.71 & 0.08 & -8.99 & 0.0000 \\ 
    Married Hispanic -  Married White & -1.21 & 0.05 & -22.92 & 0.0000 \\ 
    Married Hispanic -  Not married White & -0.45 & 0.05 & -8.57 & 0.0000 \\ 
    Not married Hispanic -  Married Others & -0.93 & 0.08 & -11.72 & 0.0000 \\ 
    Not married Hispanic -  Not married Others & -0.36 & 0.08 & -4.55 & 0.0000 \\ 
    Not married Hispanic -  Married White & -1.43 & 0.05 & -26.45 & 0.0000 \\ 
    Not married Hispanic -  Not married White & -0.68 & 0.05 & -12.54 & 0.0000 \\ 
    Married Others -  Not married Others & 0.57 & 0.07 & 8.59 & 0.0000 \\ 
    Married Others -  Married White & -0.50 & 0.06 & -7.74 & 0.0000 \\ 
    Not married Others -  Married White & -1.07 & 0.06 & -16.50 & 0.0000 \\ 
    Not married Others -  Not married White & -0.31 & 0.06 & -4.84 & 0.0000 \\ 
    Married White -  Not married White & 0.75 & 0.02 & 40.59 & 0.0000 \\ 
    Married Others -  Not married White & 0.26 & 0.06 & 4.01 & 0.0001 \\ 
    Not married Black -  Not married Hispanic & -0.16 & 0.07 & -2.38 & 0.0193 \\ 
    Married Black -  Married Hispanic & 0.16 & 0.07 & 2.14 & 0.0347 \\ 
    Married Hispanic -  Not married Others & -0.14 & 0.08 & -1.76 & 0.0820 \\ 
    Married Black -  Not married Others & 0.02 & 0.08 & 0.21 & 0.8318 \\ 
   \hline
\end{tabular}
\caption{Post-hoc test of Marital status and Ethnicity} 
\label{tab:ms_race_origin}
\end{table}

% latex table generated in R 3.3.1 by xtable 1.8-2 package
% Mon Mar 13 12:40:24 2017
\begin{table}[H]
\footnotesize
\centering
\begin{tabular}{lrrrr}
  \hline
Factor Levels & Estimate & Standard Error & t-value & p-value \\ 
  \hline
  Married High School or less -  Not married High School or less & 0.41 & 0.03 & 13.63 & 0.0000 \\ 
    Married High School or less -  Married Some college, diploma, assoc & -0.53 & 0.04 & -12.40 & 0.0000 \\ 
    Married High School or less -  Married Bachelors or higher & -1.80 & 0.05 & -35.43 & 0.0000 \\ 
    Married High School or less -  Not married Bachelors or higher & -1.16 & 0.05 & -21.21 & 0.0000 \\ 
    Not married High School or less -  Married Some college, diploma, assoc & -0.93 & 0.04 & -20.91 & 0.0000 \\ 
    Not married High School or less -  Not married Some college, diploma, assoc & -0.42 & 0.04 & -10.19 & 0.0000 \\ 
    Not married High School or less -  Married Bachelors or higher & -2.21 & 0.05 & -42.18 & 0.0000 \\ 
    Not married High School or less -  Not married Bachelors or higher & -1.57 & 0.05 & -31.25 & 0.0000 \\ 
    Married Some college, diploma, assoc -  Not married Some college, diploma, assoc & 0.52 & 0.03 & 16.60 & 0.0000 \\ 
    Married Some college, diploma, assoc -  Married Bachelors or higher & -1.28 & 0.05 & -26.12 & 0.0000 \\ 
    Married Some college, diploma, assoc -  Not married Bachelors or higher & -0.63 & 0.05 & -11.94 & 0.0000 \\ 
    Not married Some college, diploma, assoc -  Married Bachelors or higher & -1.79 & 0.05 & -34.60 & 0.0000 \\ 
    Not married Some college, diploma, assoc -  Not married Bachelors or higher & -1.15 & 0.05 & -23.79 & 0.0000 \\ 
    Married Bachelors or higher -  Not married Bachelors or higher & 0.64 & 0.03 & 18.55 & 0.0000 \\ 
    Married High School or less -  Not married Some college, diploma, assoc & -0.01 & 0.05 & -0.18 & 0.8540 \\ 
   \hline
\end{tabular}
\caption{Post-hoc test of Marital status and Education} 
\label{tab:ms_education}
\end{table}


% latex table generated in R 3.3.1 by xtable 1.8-2 package
% Mon Mar 13 12:57:48 2017
\begin{table}[H]
\footnotesize
\centering
\begin{tabular}{lrrrr}
  \hline
Factor Levels & Estimate & Standard Error & t-value & p-value \\ 
  \hline
 Male - Female & -0.03 & 0.01 & -6.56 & 0.0000 \\ 
   \hline
\end{tabular}
\caption{Post-hoc test of Gender for households with Disability} 
\label{tab:genderDisb}
\end{table}

% latex table generated in R 3.3.1 by xtable 1.8-2 package
% Mon Mar 13 12:57:59 2017
\begin{table}[H]
\footnotesize
\centering
\begin{tabular}{lrrrr}
  \hline
Factor Levels & Estimate & Standard Error & t-value & p-value \\ 
  \hline
 Married - Not married & -0.10 & 0.01 & -17.80 & 0.0000 \\ 
   \hline
\end{tabular}
\caption{Post-hoc test of Marital status for households with Disability} 
\label{tab:msDisb}
\end{table}

% latex table generated in R 3.3.1 by xtable 1.8-2 package
% Mon Mar 13 12:58:26 2017
\begin{table}[H]
\footnotesize
\centering
\begin{tabular}{lrrrr}
  \hline
Factor Levels & Estimate & Standard Error & t-value & p-value \\ 
  \hline
 Black - White & 0.05 & 0.01 & 5.54 & 0.0000 \\ 
   Others - White & 0.07 & 0.01 & 6.17 & 0.0000 \\ 
   Hispanic - White & 0.04 & 0.01 & 3.92 & 0.0002 \\ 
   Hispanic - Others & -0.03 & 0.02 & -1.80 & 0.1077 \\ 
   Black - Others & -0.02 & 0.01 & -1.14 & 0.3030 \\ 
   Black - Hispanic & 0.01 & 0.01 & 0.80 & 0.4227 \\ 
   \hline
\end{tabular}
\caption{Post-hoc test of Ethnicity for households with Disability} 
\label{tab:race_originDisb}
\end{table}

% latex table generated in R 3.3.1 by xtable 1.8-2 package
% Mon Mar 13 12:58:43 2017
\begin{table}[H]
\footnotesize
\centering
\begin{tabular}{lrrrr}
  \hline
Factor Levels & Estimate & Standard Error & t-value & p-value \\ 
  \hline
 High School or less - Some college, diploma, assoc & 0.07 & 0.01 & 9.41 & 0.0000 \\ 
   High School or less - Bachelors or higher & 0.17 & 0.01 & 17.36 & 0.0000 \\ 
   Some college, diploma, assoc - Bachelors or higher & 0.10 & 0.01 & 10.33 & 0.0000 \\ 
   \hline
\end{tabular}
\caption{Post-hoc test of Education for households with Disability} 
\label{tab:educationDisb}
\end{table}

% latex table generated in R 3.3.1 by xtable 1.8-2 package
% Mon Mar 13 12:59:09 2017
\begin{table}[H]
\footnotesize
\centering
\begin{tabular}{lrrrr}
  \hline
Factor Levels & Estimate & Standard Error & t-value & p-value \\ 
  \hline
  Male Married -  Male Not married & -0.08 & 0.01 & -11.55 & 0.0000 \\ 
    Male Married -  Female Not married & -0.14 & 0.01 & -18.46 & 0.0000 \\ 
    Female Married -  Male Not married & -0.07 & 0.01 & -8.79 & 0.0000 \\ 
    Female Married -  Female Not married & -0.12 & 0.01 & -17.79 & 0.0000 \\ 
    Male Not married -  Female Not married & -0.05 & 0.01 & -7.96 & 0.0000 \\ 
    Male Married -  Female Married & -0.01 & 0.01 & -2.23 & 0.0255 \\ 
   \hline
\end{tabular}
\caption{Post-hoc test of Gender and Marital status for households with Disability} 
\label{tab:gender:msDisb}
\end{table}

% latex table generated in R 3.3.1 by xtable 1.8-2 package
% Mon Mar 13 13:00:38 2017
\begin{table}[H]
\footnotesize
\centering
\begin{tabular}{lrrrr}
  \hline
Factor Levels & Estimate & Standard Error & t-value & p-value \\ 
  \hline
  Married Black -  Not married Black & -0.17 & 0.01 & -14.94 & 0.0000 \\ 
    Married Black -  Not married Hispanic & -0.09 & 0.02 & -5.16 & 0.0000 \\ 
    Married Black -  Not married Others & -0.16 & 0.02 & -9.22 & 0.0000 \\ 
    Married Black -  Not married White & -0.07 & 0.01 & -5.60 & 0.0000 \\ 
    Not married Black -  Married Hispanic & 0.11 & 0.02 & 7.29 & 0.0000 \\ 
    Not married Black -  Not married Hispanic & 0.08 & 0.02 & 4.94 & 0.0000 \\ 
    Not married Black -  Married Others & 0.13 & 0.02 & 8.38 & 0.0000 \\ 
    Not married Black -  Married White & 0.18 & 0.01 & 16.62 & 0.0000 \\ 
    Not married Black -  Not married White & 0.10 & 0.01 & 8.84 & 0.0000 \\ 
    Married Hispanic -  Not married Others & -0.11 & 0.02 & -6.10 & 0.0000 \\ 
    Married Hispanic -  Married White & 0.07 & 0.01 & 5.72 & 0.0000 \\ 
    Not married Hispanic -  Not married Others & -0.08 & 0.02 & -4.17 & 0.0000 \\ 
    Not married Hispanic -  Married White & 0.10 & 0.01 & 7.79 & 0.0000 \\ 
    Married Others -  Not married Others & -0.13 & 0.02 & -8.57 & 0.0000 \\ 
    Not married Others -  Married White & 0.18 & 0.01 & 12.26 & 0.0000 \\ 
    Not married Others -  Not married White & 0.10 & 0.01 & 6.50 & 0.0000 \\ 
    Married White -  Not married White & -0.08 & 0.01 & -16.53 & 0.0000 \\ 
    Married Others -  Married White & 0.05 & 0.01 & 3.46 & 0.0008 \\ 
    Married Black -  Married Hispanic & -0.05 & 0.02 & -3.39 & 0.0010 \\ 
    Not married Hispanic -  Married Others & 0.06 & 0.02 & 3.13 & 0.0024 \\ 
    Married Hispanic -  Not married Hispanic & -0.03 & 0.01 & -3.01 & 0.0035 \\ 
    Married Others -  Not married White & -0.04 & 0.01 & -2.77 & 0.0073 \\ 
    Married Black -  Married Others & -0.03 & 0.02 & -1.92 & 0.0666 \\ 
    Not married Hispanic -  Not married White & 0.02 & 0.01 & 1.38 & 0.1941 \\ 
    Married Hispanic -  Married Others & 0.02 & 0.02 & 1.33 & 0.2070 \\ 
    Married Hispanic -  Not married White & -0.01 & 0.01 & -1.21 & 0.2420 \\ 
    Married Black -  Married White & 0.01 & 0.01 & 1.15 & 0.2606 \\ 
    Not married Black -  Not married Others & 0.00 & 0.02 & 0.02 & 0.9807 \\ 
   \hline
\end{tabular}
\caption{Post-hoc test of Marital status and Ethnicity for households with Disability} 
\label{tab:ms:race_originDisb}
\end{table}

% latex table generated in R 3.3.1 by xtable 1.8-2 package
% Mon Mar 13 13:01:30 2017
\begin{table}[H]
\footnotesize
\centering
\begin{tabular}{lrrrr}
  \hline
Factor Levels & Estimate & Standard Error & t-value & p-value \\ 
  \hline
  Married High School or less -  Not married High School or less & -0.13 & 0.01 & -18.85 & 0.0000 \\ 
    Married High School or less -  Married Some college, diploma, assoc & 0.06 & 0.01 & 6.80 & 0.0000 \\ 
    Married High School or less -  Not married Some college, diploma, assoc & -0.05 & 0.01 & -5.05 & 0.0000 \\ 
    Married High School or less -  Married Bachelors or higher & 0.14 & 0.01 & 12.63 & 0.0000 \\ 
    Married High School or less -  Not married Bachelors or higher & 0.07 & 0.01 & 6.05 & 0.0000 \\ 
    Not married High School or less -  Married Some college, diploma, assoc & 0.19 & 0.01 & 20.38 & 0.0000 \\ 
    Not married High School or less -  Not married Some college, diploma, assoc & 0.08 & 0.01 & 9.64 & 0.0000 \\ 
    Not married High School or less -  Married Bachelors or higher & 0.27 & 0.01 & 23.58 & 0.0000 \\ 
    Not married High School or less -  Not married Bachelors or higher & 0.21 & 0.01 & 18.17 & 0.0000 \\ 
    Married Some college, diploma, assoc -  Not married Some college, diploma, assoc & -0.11 & 0.01 & -14.49 & 0.0000 \\ 
    Married Some college, diploma, assoc -  Married Bachelors or higher & 0.08 & 0.01 & 7.16 & 0.0000 \\ 
    Not married Some college, diploma, assoc -  Married Bachelors or higher & 0.19 & 0.01 & 16.19 & 0.0000 \\ 
    Not married Some college, diploma, assoc -  Not married Bachelors or higher & 0.12 & 0.01 & 11.17 & 0.0000 \\ 
    Married Bachelors or higher -  Not married Bachelors or higher & -0.07 & 0.01 & -7.06 & 0.0000 \\ 
    Married Some college, diploma, assoc -  Not married Bachelors or higher & 0.01 & 0.01 & 1.10 & 0.2723 \\ 
   \hline
\end{tabular}
\caption{Post-hoc test of Marital status and Education for households with Disability} 
\label{tab:ms:educationDisb}
\end{table}




\newpage
\bibliography{bibTex_Reference}
\end{document}
