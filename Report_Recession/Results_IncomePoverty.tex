\documentclass[11pt]{extarticle} %extarticle for fontsizes other than 10, 11 And 12
%\documentclass[11p]{article}

%%%%%%%%%%%%%%%%%%%%%%%%%%%%%%%%%%%%%%%%%%%%%%%%%%%%%%%%%%%%%%%%%%%%%%
%% Input header file 
%%%%%%%%%%%%%%%%%%%%%%%%%%%%%%%%%%%%%%%%%%%%%%%%%%%%%%%%%%%%%%%%%%%%%%
%%%%%%%%%%%%%%%%%%%%%%%% Packages %%%%%%%%%%%%%%%%%%%%%%%%
\usepackage{amscd}
\usepackage{amsmath}
\usepackage{amssymb}
\usepackage{amsthm}
\usepackage{amsxtra}
\usepackage{animate}
\usepackage{bbold}
%\usepackage{bigints}
\usepackage{caption}    %% For multiple line captions
\usepackage{color, colortbl}
\usepackage{dsfont}
\usepackage{enumerate}
\usepackage[mathscr]{eucal}
%\usepackage{fancyhdr}
\usepackage{float}
%\usepackage{fullpage}  %% Dont use this for beamer presentations
\usepackage{geometry}
\usepackage{graphicx}
\usepackage{hyperref}
\usepackage{indentfirst}
\usepackage{latexsym}
\usepackage{listings}
\usepackage{longtable}  %% to add pagebreaks in between table
\usepackage{lscape}
\usepackage{mathtools}
\usepackage{microtype}
\usepackage{multirow}
\usepackage{natbib}
\usepackage{pdfpages}
\usepackage{setspace}   %% Allows to set double or single space
\usepackage{tcolorbox}  %% For colored textboxes
\usepackage{verbatim}
\usepackage{wrapfig}
\usepackage{xargs}
\usepackage{xcolor}
\DeclareGraphicsExtensions{.pdf,.png,.jpg, .jpeg}
\definecolor{LightCyan}{rgb}{0.88,1,1}

\usepackage{array}
\newcolumntype{C}[1]{>{\centering\arraybackslash}p{#1}}  %% For wrapping text in table headers

%%%%%%%%%%%%%%%%%%%%%%%% Commands %%%%%%%%%%%%%%%%%%%%%%%%
\newcommand{\Sup}{\textsuperscript}
\newcommand{\Exp}{\mathds{E}}
\newcommand{\Prob}{\mathds{P}}
\newcommand{\Z}{\mathds{Z}}
\newcommand{\Ind}{\mathds{1}}
\newcommand{\A}{\mathcal{A}}
\newcommand{\F}{\mathcal{F}}
%\newcommand{\G}{\mathcal{G}}
\newcommand{\I}{\mathcal{I}}
\newcommand{\R}{\mathcal{R}}
\newcommand{\Y}{\mathcal{Y}}
\newcommand{\Real}{\mathbb{R}}
\newcommand{\be}{\begin{equation}}
\newcommand{\ee}{\end{equation}}
\newcommand{\bes}{\begin{equation*}}
\newcommand{\ees}{\end{equation*}}
\newcommand{\union}{\bigcup}
\newcommand{\intersect}{\bigcap}
\newcommand{\Ybar}{\overline{Y}}
\newcommand{\ybar}{\bar{y}}
\newcommand{\Xbar}{\overline{X}}
\newcommand{\xbar}{\bar{x}}
\newcommand{\betahat}{\hat{\beta}}
\newcommand{\Yhat}{\widehat{Y}}
\newcommand{\yhat}{\hat{y}}
\newcommand{\Xhat}{\widehat{X}}
\newcommand{\xhat}{\hat{x}}
\newcommand{\E}[1]{\operatorname{E}\left[ #1 \right]}
%\newcommand{\Var}[1]{\operatorname{Var}\left( #1 \right)}
\newcommand{\Var}{\operatorname{Var}}
\newcommand{\Cov}[2]{\operatorname{Cov}\left( #1,#2 \right)}
\newcommand{\N}[2][1=\mu, 2=\sigma^2]{\operatorname{N}\left( #1,#2 \right)}
\newcommand{\bp}[1]{\left( #1 \right)}
\newcommand{\bsb}[1]{\left[ #1 \right]}
\newcommand{\bcb}[1]{\left\{ #1 \right\}}
\newcommand*{\permcomb}[4][0mu]{{{}^{#3}\mkern#1#2_{#4}}}
\newcommand*{\perm}[1][-3mu]{\permcomb[#1]{P}}
\newcommand*{\comb}[1][-1mu]{\permcomb[#1]{C}}
\newcommand{\indep}{\rotatebox[origin=c]{90}{$\models$}}

\DeclareMathOperator*{\argmin}{arg\,min}

\usepackage[font=small]{caption}

\geometry{left=1in, right=1in, top=1in, bottom=1in}

\begin{document}
\doublespacing
%\SweaveOpts{concordance=TRUE}
%\bibliographystyle{plain}  %Choose a bibliograhpic style
\bibliographystyle{chicago}

\title{Effects of Great Recession on Income Poverty}
\author{Subharati Ghosh, Subhrangshu Nandi, Susan Parish \\
%  Statistics PhD Student, \\
%  Research Assistant,
%  Laboratory of Molecular and Computational Genomics, \\
%  University of Wisconsin - Madison}
%\date{February 16, 2015}
\date{}
}

\maketitle
\section*{INTRODUCTION}
\noindent
{\hl{Insert Introduction}}.

\subsubsection*{Study Aim 1}
Aim 1 of this study was to analyze whether households with working age adults with disability differed from households with no working age adults with disability, during the {\emph{great recession}} \footnote{Business Cycle Dating Committee, National Bureau of Economic Research (NBER)}, using ``Income Poverty'' as a measure of economic wellbeing, controlling for demographic factors such as gender, marital status, education, race and ethnicity. 
\begin{itemize}
\item Hypothesis 1: It was hypothesized that households with working age adults with disability experienced worse ``income poverty'' levels through the great recession, controlling for demographic factors.
\item Hypothesis 2: It was hypothesized that households with working age adults with disability experienced worse declines in income poverty levels during the great recession, controlling for demographic factors.
\end{itemize}

\subsubsection*{Study Aim 2}
Aim 2 of this study was to analyze how demographic factors such as gender, marital status, education, race/ethnicity were associated with economic wellbeing of households with a working age adult with disability, during the {\emph{great recession}}. 

\section*{BACKGROUND}
\noindent
{\hl{Insert Literature review and background}}.

\section*{DATA AND METHOD}
\subsubsection*{DATA}
Data was drawn from the Survey of Income and Program Participation (SIPP) 2008 panel. SIPP is administered by the US Census Bureau ({\footnote{For more information on the SIPP 2008 panel schedule, please refer to this \href{http://www.census.gov/programs-surveys/sipp/data/2008-panel.html}{US Census Bureau website}}}), and is representative of non-institutionalized US households. The SIPP 2008 panel started from July 2008 and lasted till June 2013, including a total of 13 waves. The waves overlapped with twelve of the eighteen months {\footnote{\href{http://www.nber.org/cycles/}{NBER Recession Cycles}}} of the “Great recession” and its long wake. Households selected, were followed through the entire panel, and were interviewed very fourth month on a set of core questions, which inquired on household demographics, labor-force participation, participation in the various safety-net programs, asset ownership over the last three months. In addition to the core questions, the SIPP also administered specific modules or topical questions, asked only once during the entire study panel. The topical modules varied by the waves and included questions on marital history, disability, material hardships, assets-liabilities etc. The reference period for the modules varied. Of interest to this study was wave 6 of the 2008 panel, which specifically enquired on adult disability status.

To be included in the study sample, respondents had to meet a set of criteria. First, it was necessary for households to have at least participated in wave six of the study, the wave that had a specific module enquiring on adult disability. Second, the reference persons of the households had to be same in all the waves the households participated in. And third, the reference person of the selected households had to be adults (18 years and older) throughout the household's participation in the study. Based on the inclusion criteria, a total of 33,547 households met the inclusion criteria.

\subsubsection*{MEASURES}
\noindent
{\emph{Dependent variable}}\\
The ratio of average quarterly household income and federal poverty level (100\% FPL) was chosen as a measure of income poverty and was used as the dependent variable in our analysis. We named the dependent variable FPL100-ratio. An FPL100-ratio lower than one in any quarter indicated the household was below 100\% of the federal poverty level in that quarter. In the sample, the quarterly income data ranged from -\$27,180 to \$108,900, the average being \$5240 and median \$3,874. The negative incomes were associated with households owning business that incurred losses in those quarters. The FPL100-ratio ranged from -17.95 to 89.48, with the average being 3.817 and the median 2.924. 

\noindent
{\emph{Key Predictors}}\\
There were two key predictors in our analysis: {\emph{time}} and {\emph{adult disability}}. WHY QUARTERS? Time (quarters) was treated was a continuous variable. Monthly  adult disability was treated as a dichotomous factor and the interaction between time and adult disability was treated as a continuous variable. Wave 6 of the 2008 panel included detailed questions to assess adult disability. Adult Disability was assessed by asking the household reference person, whether there was any adult in the household who had experienced difficulties with activities of daily living, or have been using assistive devices, such as canes, or had mental retardation, learning disability, developmental disability, or Alzheimer ’ s or any disease that impacted memory, resulting in loss of memory, forgetfulness . Once identified, only those households were selected where the person with disability was between ages 18 and 64, with a chronic illness/ disability, i.e. of the duration of at least one year, to capture the severity of the disease or illness. Based on the criteria , a total of 7,443 households (22.16\%) identified having least one working age adult with disability. A dichotomous variable indicated whether the household had a working age adult with disability or whether it did not. 

\noindent
{\emph{Control variables}}\\
We controlled for four demographic factors: {\emph{gender}}, {\emph{marital status}}, {\emph{education}}, {\emph{race}}, and {\emph{ethnicity}} of the reference persons of the households in our analysis. Gender was a dichotomous variable (male \& female), and so was marital status. Household reference persons who were divorced, widowed and never married were categorized as `not
married'. Education had three categories, `high-school or less', `some college, diploma, associated degrees' and `bachelors or higher'. We also included a variable labeled race/ethnicity, which was
based on two variables, ethnicity and racial origin. The SIPP assessed ethnicity using a dichotomous variable, which assessed whether the householder was or was not of Hispanic or
Latino origin. Racial origin was assessed by asking respondents to identify themselves as `White alone', `Black alone', `Asian' and `Others', including Native Hawaiian and Pacific Islanders. Asians
and Others were collapsed into one category. Taking the two measures, racial origin and ethnicity, resulted in a measure race/ethnicity, which had four categories `non-hispanic white', `non-hispanic black', `hispanic' and `others'. Non-Hispanic White and non-Hispanic Black are henceforth referred to as `Black' and `White'. 

\subsubsection*{ANALYTIC STRATEGY}
A mixed (fixed and random) effects model was fit to analyze how households with a working age adult with disability differ from households with no working age adult with disability, during the great recession, using “Income Poverty” as a measure of economic wellbeing, controlling for demographic factors. Since this dataset is longitudinal in nature, to account for ``between household'' differences a mixed effect model was used. $Y$ denoted the vector of responses (FPL100-ratio). $\Theta$ denoted the vector of fixed effect factors like gender, marital status, education level, race/ethnicity of reference person, along with their interactions. $\beta$ denoted another fixed effect of time (in quarters), starting from 2008-Q3 and ending in 2013-Q1. $b$ denoted the household level random effect (random intercept). The separate estimation of $b_i$ from $\epsilon_{ij}$ ensures the separate estimation of the two types of variability (between household, $b_i$, and within household, $\epsilon_{ij}$). For example, a simple mixed-effects model for the analysis could be written as
\vspace{-0.5cm}
\begin{equation}
Y_{ij} = \beta_0 + \beta t_j + X_i\Theta + b_i + \epsilon_{ij}
\label{eq:MixedEffects1}
\end{equation}
where, $\epsilon_{ij}$ are measurement errors, $i$ ranges from $1$ to $H$, the number of households, $j$ from $1$ to $T$, the total number of quarters. In this model, the response from the $i^{th}$ household at time $t_j$ is assumed to differ from the population mean $\beta_0 + \beta t_j + X_i\Theta$ by a household effect $b_i$ and a within household measurement error $\epsilon_{ij}$. The within household and between household errors are assumed to be normal and independent \footnote{($b_i \sim \N(0, \sigma^2_b),\ \ \epsilon_{ij} \sim \N(0, \sigma^2), \ \ b_i \indep \epsilon_{ij}, \forall i, j$)}. ``time'' was a fixed effect. 

The  trough, of the great recession was reached in the second quarter of 2009 (marking the technical end of the recession, defined as at least two consecutive quarters of declining GDP) \footnote{Business Cycle Dating Committee, National Bureau of Economic Research (NBER)}. According to NBER, June 2009 was the final month of the recession. We checked if this was reflected in the FPL100-ratio as a downward trend in the initial quarters followed by an upward trend. A linear term in ``time'' was insufficient to capture this effect. We added a second order term $\text{time}^2$ to test the change in direction of trend. The second order term was added after centering the original ``time'' variable, to avoid introducing multicollinearity. An indicator variable was used to denote the presence of working age adult with disability in a household. An interaction term between this indicator variable and time was also included to estimate the difference in slopes between households with and without a working age adult with disability. Below is the final model that was fit: 
\vspace{-0.5cm}
\begin{equation}
Y_{ij} = \beta_0 + \beta_1 t_j + \beta_2 t_j^2 + \beta_D \Ind_{D_i} + \beta_t(\Ind_{D_i}*t_j) + X_i\Theta + b_i + \epsilon_{ij}
\label{eq:MixedEffects2}
\end{equation}

where, $\Ind_{D_i} = 1$, if household $i$ has a working age adult with disability, else $\Ind_{D_i} = 0$. To test the hypotheses in aim 1 we tested the significance of $\beta_D$ (for hypothesis 1) and $\beta_t$ (for hypothesis 2). Significance of coefficients $\beta_1$ and $\beta_2$ were tested to analyze the overall trends of FPL100-ratio over the study period. In addition, interactions between demographic factors, and between disability and demograhic factors were also tested. The demographic factors were considered as fixed effects.

The final model was fit with some of the fixed effect factors along with their interactions after performing ``backward elimination'' on the full model. Elimination of the fixed effects were done by the principle of marginality, that is: the highest order interactions are tested first: if they are significant, the lower order effects were included in the model without testing for significance. The p-values for the fixed effects are estimated from the F statistics, with ``Satterthwaite'' approximation (\cite{Satterthwaite_1946_Biometrics}) denominator degrees of freedom. The p-values for the random effect were computed from likelihood ratio tests (\cite{Morrell_1998_Biometrics}). 

\noindent
{\bf{Post-hoc tests}} \\
Post-hoc tests were conducted between categories of all demographic factors and their interactions, by calculating differences of ``Least Squares Means'' using R package ``lmerTest'' (\cite{Kuznetsova_etal_2015_R-lmerTest}), with ``Satterthwaite'' approximation (\cite{Satterthwaite_1946_Biometrics}) of the denominator degrees of freedom.

\noindent
{\bf{Multiple testing correction}}\\
When conducting post-hoc tests for demographic factors and their interactions, due to multiple categories of these factors the size of the tests could be inflated. Sequentially rejective {\emph{Bonferroni procedure}} (\cite{Holm_1979_SJS}) and {\emph{Benjamini-Hochberg procedure}} (\cite{Benjamini_Hochberg_1995_JRSSB}) remain the two most popular multiple testing correction procedures. Holm's sequentially rejective Bonferroni procedure controls the family-wise type-I error rate (FWER) and is more powerful than the classical Bonferroni procedure. Benjamini-Hochberg controls the false discovery rate (FDR) which is the expected value of false discovery proportion. Controlling FWER usually proves to be too conservative. Hence, we used the Benjamini-Hochberg procedure, which is less conservative, but more powerful than Bonferroni correction. All post-hoc test p-values reported were Benjamini-Hochberg corrected.

\noindent
{\bf{Computational software}}\\
All analysis were conducted using the statistical software R (\cite{R}), version 3.3.1. The mixed effects models were fit using the R-package ``lme4'' (\cite{R-lme4}) and all hypothesis tests were done using the R package ``lmerTest'' (\cite{Kuznetsova_etal_2015_R-lmerTest}). 

\section*{RESULTS}
Table \ref{tab:DescStats} illustrates the descriptive statistics of the sample. As noted earlier, not all households participated in all waves of the survey, other than wave six, which included questions on disability status. This table is a descriptive statistics of the sample during wave six. {\hl{Add more writeup about descriptive statistics}}

\noindent
{\bf{Aim 1}}\\
Aim 1 of this study was to analyze how households with a working age adult with disability differed from households with no working age adult with disability, during the {\emph{great recession}}, using ``Income Poverty'' as a measure of economic wellbeing, controlling for demographic factors such as gender, marital status, education, race/ethnicity. Table \ref{tab:Table2Reg} shows the mixed-effects regression results with FPL100-ratio (measure of income poverty) as the dependent variable, time and disability status as the key predictors and demographic factors as the control variables, as explained in equation \ref{eq:MixedEffects2}. Results showed that households with working age adults with disability, on an average experienced significantly worse income poverty (FPL100-ratio, $\beta_D = -0.725, p < 0.001$) compared with households with no working age adults with disability. This is also illustrated in figure \ref{fig:disability}, which shows the trends of FPL100-ratios of households with and without a working age adults with disability. 

We could not prove hypothesis 2 of aim 1. Results showed that the trends of income poverty were not significantly different between households with and without working age adults with disability, over the study period ($\beta_t = 0.015, p = 0.9994$). This is evident in figure \ref{fig:disability} as well. 

The regression results in table \ref{tab:Table2Reg} showed that FPL100-ratio, on an average, {\bf{decreased}} by $0.054$ every year ($\beta_1 = -0.054, p < 0.001$). The coefficient of the quadratic term of Time ($\beta_2 = 0.0073, p < 0.01$) indicated rate of change of slope was positive. In other words, although the FPL100-ratio decreased over time (as $\beta_1 < 0$), it flattened out and started increasing, in the latter parts of the study period. 

As shown in table \ref{tab:Table2Reg}, gender was statistically significant; households with ``female'' reference persons had on an average 0.368 lower FPL100-ratio ($\beta = -0.368, p < 0.001$) compared to households with ``male'' reference persons. Marital status was statistically significant; households with ``not-married'' reference persons had on an average 0.611 lower FPL100-ratio ($\beta = -0.611, p < 0.001$) compared to households with ``married'' reference person. race/ethnicity were statistically significant; households with ``black'' reference persons had on an average 1.28 lower FPL100-ratio ($\beta = -1.284, p < 0.001$) than households with ``white'' reference person. The difference was even more stark between households with hispanic and white reference persons ($\beta = -1.505, p < 0.001$). Education was also statistically significant; households with reference persons with ``high school or less'' education levels had on an average 2.204 ($\beta = -2.204, p < 0.001$) lower FPL100-ratio than households with reference persons with ``bachelors or higher'' education levels. All significant interactions between disability and demographic factors and between different demographic factors themselves are displayed in table \ref{tab:Table2Reg}.

\noindent
{\bf{Aim 2}}\\
Aim 2 of the study was to analyze how demographic factors such as gender, marital status, education, race/ethnicity were associated with economic wellbeing (measured by FPL100-ratio) of households {\underline{with}} a working age adult with disability, during the study period. A separate mixed effects model was fit with the same demographic factors, on households with working age adults with disability. Table \ref{tab:Table3Reg} shows the mixed-effects regression results with FPL100-ratio as the dependent variable, time as the key predictor and demographic factors as the control variables. Several post-hoc tests were conducted on factors with multiple levels and their interactions. Table \ref{tab:Table3PostHoc} lists the post-hoc test results. Figure \ref{fig:Disab_Demographics} displays the trends of the different subgroups of the data, separated by demographic factors. 

Results from table \ref{tab:Table3Reg} shows that gender, marital status, education, race/ethnicity and some of their interactions have statistically significant associations with FPL100-ratio for households with disability during the study period. This is also illustrated in figures \ref{fig:disab_gender} for gender, \ref{fig:disab_MS} for marital status, \ref{fig:disab_education} for education and \ref{fig:disab_race} for race/ethnicity. 

Figure \ref{fig:disab_race} illustrates that households with ``hispanic'' reference persons had minimum FPL100-ratios throughout the study period. Another important observation is the different shapes of the FPL100-ratios of the four races. Households with ``white'' reference persons had a gradual and steady incline in their average FPL100-ratios after 2011. However, this behavior was not observed in households with ``black'', ``hispanic'' or ``others'' heads. In figure \ref{fig:disab_education}, households where the education level of their heads were ``high school or less'' experienced a decline in their FPL100-ratios, just like the other groups, but never experienced any improvement in the latter parts of the study. 

The association of marital status with FPL100-ratios between the two models is worth highlighting. The association is almost double in households with disability ($\beta = -0.611$ in table \ref{tab:Table2Reg} and $\beta=-1.119$ in table \ref{tab:Table3Reg}). 

Figure \ref{fig:contrasts} illustrates the trends of FPL100-ratios of two contrasting household types: one with white, married, male (with education bachelors or higher) as reference persons, the other with not married, black, female (with high school or less education). 

\section*{Discussion}
The FPL100-ratios decline sharply between 2008 and 2010, flatten out and then increase gradually after 2011. The quadratic term of Time in the model captures this behavior. It is noticable that the decline in FPL100-ratios was sharper than the gradual incline that followed. A similar behavior is observed in both types of households. 

The positive sign of $\beta_t$ indicated that the downward trend of households with disability was not as steep as the households with no disability. This apparently contradictory finding led us to conclude that households with disability although had ``significantly'' worse FPL100-ratio throughout the study period, the households without disability experienced more severe declines in their FPL100-ratios. This could throw some light on the impact of different supplementary coverage programs on households with disability.

\section*{Limitations}
\begin{enumerate}
\item Although a linear mixed effects regression model discovered some conventional and some interesting patterns in the relationships between response and demographic factors, along with disability, the trajectory of income poverty over the study period for some households were not linear. This modeling approach did not capture trajectory shapes of individual households. A non-parametric fitting of the income poverty trajectories could be tried as a pre-processing step before testing for differences in behavior between different groups of households. 

\item Some households in the sample did not participate over all the waves. Since households that participated in wave six were included there were some households that were first interviewed in wave six and some that were no longer interviewed after wave six. There were no means of determining the reasons for dropping out from the survey, nor the reasons for late inception into the survey. Since the {\emph{great recession}} was a significant economic and social event, we included households without complete participation in order to maximize the sample size, and incorporate the effect of the recession on more households. If, however, the reasons for dropping out or late joining had an association with the outcome of the study (income poverty), including those households could increase bias in the estimates, in spite of the estimates being more stable (less variance). Chapter 2 in SIPP users guide \footnote{https://www2.census.gov/programs-surveys/sipp/guidance/SIPP\_2008\_USERS\_Guide\_Chapter2.pdf} mentions that the survey weights are adjusted to account for some types of household nonresponse with the objective of ameliorating the nonresponse bias. 
\end{enumerate}



\newpage
%% Figures for Income Poverty Report

\section{Figures}

\begin{figure}[H]
\caption{Average fitted values of FPL100-ratios, by Disability status}
\centering
\includegraphics[scale=0.85]{../Plots/PredictedFPLPlot_Disability.pdf}
\label{fig:disability}
\end{figure}

\begin{figure}[H]
\caption{Average fitted values of FPL100-ratios, by Gender, for households with a working age adult with disability}
\centering
\includegraphics[scale=0.75]{../Plots/PredictedFPLPlot_Gender.pdf}
\label{fig:gender}
\end{figure}

\begin{figure}[H]
\caption{Average fitted values of FPL100-ratios, by Marital status, for households with a working age adult with disability}
\centering
\includegraphics[scale=0.75]{../Plots/PredictedFPLPlot_MS.pdf}
\label{fig:MS}
\end{figure}

\begin{figure}[H]
\caption{Average fitted values of FPL100-ratios, by Race/Ethnicity, for households with a working age adult with disability}
\centering
\includegraphics[scale=0.75]{../Plots/PredictedFPLPlot_Ethnicity.pdf}
\label{fig:race}
\end{figure}

\begin{figure}[H]
\caption{Average fitted values of FPL100-ratios, by Education, for households with a working age adult with disability}
\centering
\includegraphics[scale=0.75]{../Plots/PredictedFPLPlot_Education.pdf}
\label{fig:education}
\end{figure}

\begin{figure}[H]
\caption{Average fitted values of FPL100-ratios, for two contrasting household types}
\centering
\includegraphics[scale=0.85]{../Plots/PredictedFPLPlot_Contrasts.pdf}
\label{fig:contrasts}
\end{figure}


%% Tables for Income Poverty Report

\section{Tables}

\noindent
\begin{table}[H] 
\centering 
\footnotesize
%\begin{tabular}{@{\extracolsep{5pt}}lccc|c}
\begin{tabular}{l|c}
\hline 
\hline 
% & \multicolumn{4}{c}{\textit{Dependent variable:}} \\ 
%& Model1 (with Baseline) & Model2 (with No Baseline) \\
%\hline 
%Intercept		&	$2.779^{***}$		&	$0.1478^{**}$ 		\\
%			&	$(0.0706)$		&	$(0.0646)$		\\
%time (Year-Quarter)	&	$-0.0553^{***}$		&	$-0.0543^{***}$ 	\\
%			&	$(0.0021)$		&	$(0.0021)$		\\
%$\text{time}^2$		&	$0.0073^{***}$		&	$0.0075^{***}$ 		\\
%			&	$(0.0013)$		&	$(0.0013)$		\\
%Disability		&	$-0.5061^{***}$		&	$-0.0233$		\\    
%			&	$(0.0564)$		&	$(0.0499)$		\\
%Year-Quarter*Disability	&	$0.0150^{***}$		&	$0.0150^{***}$		\\
%			&	$(0.0043)$		&	$(0.0043)$		\\
%
Continuous Indep Variables & Model coefficients \\
\hline 
Intercept		&	$2.779^{***}$		\\
			&	$(0.0706)$		\\
time (Year-Quarter)	&	$-0.0553^{***}$		\\
			&	$(0.0021)$		\\
$\text{time}^2$		&	$0.0073^{***}$		\\
			&	$(0.0013)$		\\
Disability		&	$-0.5061^{***}$		\\    
			&	$(0.0564)$		\\
Year-Quarter*Disability	&	$0.0150^{***}$		\\
			&	$(0.0043)$		\\
\hline 
\hline 
\multicolumn{2}{r}{\textit{Note:}  $^{*}$p$<$0.1; $^{**}$p$<$0.05; $^{***}$p$<$0.01} \\ 

\end{tabular}
\caption{Coefficients of Fixed effects in regression models} 
\label{tab:FixedEffectsBetas} 
\end{table}


% latex table generated in R 3.3.1 by xtable 1.8-2 package
% Mon Mar 13 12:08:48 2017
\begin{table}[H]
\footnotesize
\centering
\begin{tabular}{lrrrrr}
  \hline
  & Sum Sq & Mean Sq & NumDF & F.value & p.value \\ 
  \hline
  Time & 5380.93 & 5380.93 & 1 & 446.58 & 0.0000 \\ 
  Disability & 2280.29 & 2280.29 & 1 & 189.25 & 0.0000 \\ 
  Gender & 3143.82 & 3143.82 & 1 & 260.92 & 0.0000 \\ 
  Marital status & 5788.34 & 5788.34 & 1 & 480.40 & 0.0000 \\ 
  Ethnicity & 8743.91 & 2914.64 & 3 & 241.90 & 0.0000 \\ 
  Education & 15873.19 & 7936.60 & 2 & 658.69 & 0.0000 \\ 
  Gender \& Marital status & 3184.05 & 3184.05 & 1 & 264.26 & 0.0000 \\ 
  Marital status \& Ethnicity & 1670.73 & 556.91 & 3 & 46.22 & 0.0000 \\ 
  Ethnicity \& Education & 652.73 & 108.79 & 6 & 9.03 & 0.0000 \\ 
  $\text{Time}^2$ & 414.33 & 414.33 & 1 & 34.39 & 0.0000 \\ 
  Marital status \& Education & 450.29 & 225.15 & 2 & 18.69 & 0.0000 \\ 
  Gender \& Education & 310.12 & 155.06 & 2 & 12.87 & 0.0000 \\ 
  Disability \& Gender & 182.78 & 182.78 & 1 & 15.17 & 0.0001 \\ 
  Time:adult\_disb & 143.66 & 143.66 & 1 & 11.92 & 0.0006 \\ 
  Disability \& Education & 116.32 & 58.16 & 2 & 4.83 & 0.0080 \\ 
   \hline
\end{tabular}
\caption{FPL100 vs demographic factors and time and disability status} 
\label{tab:Anova1}
\end{table}


% latex table generated in R 3.3.1 by xtable 1.8-2 package
% Mon Mar 13 12:10:56 2017
\begin{table}[H]
\footnotesize
\centering
\begin{tabular}{lrrrrr}
  \hline
  & Sum Sq & Mean Sq & NumDF & F.value & p.value \\ 
  \hline
  Marital status & 66.83 & 66.83 & 1 & 316.81 & 0.0000 \\ 
  Education & 65.21 & 32.60 & 2 & 154.56 & 0.0000 \\ 
  Marital status \& Ethnicity & 18.18 & 6.06 & 3 & 28.73 & 0.0000 \\ 
  Ethnicity & 14.34 & 4.78 & 3 & 22.66 & 0.0000 \\ 
  Gender & 9.09 & 9.09 & 1 & 43.08 & 0.0000 \\ 
  Ethnicity \& Education & 11.36 & 1.89 & 6 & 8.98 & 0.0000 \\ 
  Marital status \& Education & 8.03 & 4.01 & 2 & 19.03 & 0.0000 \\ 
  Gender \& Marital status & 4.40 & 4.40 & 1 & 20.86 & 0.0000 \\ 
  Time & 3.39 & 3.39 & 1 & 16.08 & 0.0001 \\ 
  $\text{Time}^2$ & 0.06 & 0.06 & 1 & 0.26 & 0.6074 \\ 
  \hline
\end{tabular}
\caption{FPL100 vs demographic factors and time, for households with Disability} 
\label{tab:Anova2}
\end{table}


% latex table generated in R 3.3.1 by xtable 1.8-2 package
% Mon Mar 13 12:17:15 2017
\begin{table}[H]
\footnotesize
\centering
\begin{tabular}{lrrrrr}
  \hline
  & Sum Sq & Mean Sq & NumDF & F.value & p.value \\ 
  \hline
  Time & 13.16 & 13.16 & 1 & 86.80 & 0.0000 \\ 
  Gender & 48.49 & 48.49 & 1 & 319.96 & 0.0000 \\ 
  Marital status & 76.70 & 76.70 & 1 & 506.08 & 0.0000 \\ 
  Ethnicity & 99.18 & 33.06 & 3 & 218.15 & 0.0000 \\ 
  Education & 106.34 & 53.17 & 2 & 350.84 & 0.0000 \\ 
  Gender \& Marital status & 12.85 & 12.85 & 1 & 84.80 & 0.0000 \\ 
  Marital status \& Education & 27.95 & 13.97 & 2 & 92.21 & 0.0000 \\ 
  Ethnicity \& Education & 25.03 & 4.17 & 6 & 27.52 & 0.0000 \\ 
  Marital status \& Ethnicity & 9.60 & 3.20 & 3 & 21.12 & 0.0000 \\ 
  I(Time\verb|^|2) & 0.03 & 0.03 & 1 & 0.21 & 0.6489 \\ 
   \hline
\end{tabular}
\caption{FPL100 vs demographic factors and time, for households with No Disability} 
\label{tab:Anova3}
\end{table}

% latex table generated in R 3.3.1 by xtable 1.8-2 package
% Mon Mar 13 12:19:44 2017
\begin{table}[H]
\footnotesize
\centering
\begin{tabular}{lrrrr}
  \hline
  Factor Levels & Estimate & Standard Error & t-value & p-value \\ 
  \hline
   Black - Others & -0.54 & 0.07 & -8.02 & 0.0000 \\ 
   Black - White & -0.94 & 0.05 & -20.16 & 0.0000 \\ 
   Hispanic - Others & -0.54 & 0.07 & -7.78 & 0.0000 \\ 
   Hispanic - White & -0.94 & 0.05 & -19.60 & 0.0000 \\ 
   Others - White & -0.40 & 0.05 & -7.43 & 0.0000 \\ 
   Black - Hispanic & -0.00 & 0.06 & -0.05 & 0.9611 \\ 
  \hline
\end{tabular}
\caption{Post-hoc test of Ethnicity} 
\label{tab:race_origin}
\end{table}

% latex table generated in R 3.3.1 by xtable 1.8-2 package
% Mon Mar 13 12:21:17 2017
\begin{table}[H]
\footnotesize
\centering
\begin{tabular}{lrrrr}
  \hline
Factor Levels & Estimate & Standard Error & t-value & p-value \\ 
  \hline
 High School or less - Some college, diploma, assoc & -0.47 & 0.04 & -12.47 & 0.0000 \\ 
   High School or less - Bachelors or higher & -1.69 & 0.05 & -36.04 & 0.0000 \\ 
   Some college, diploma, assoc - Bachelors or higher & -1.21 & 0.04 & -27.03 & 0.0000 \\ 
   \hline
\end{tabular}
\caption{Post-hoc test of Education} 
\label{tab:education}
\end{table}

% latex table generated in R 3.3.1 by xtable 1.8-2 package
% Mon Mar 13 12:23:35 2017
\begin{table}[H]
\footnotesize
\centering
\begin{tabular}{lrrrr}
  \hline
Factor Levels & Estimate & Standard Error & t-value & p-value \\ 
  \hline
  Male Married -  Female Married & 0.15 & 0.03 & 5.03 & 0.0000 \\ 
    Male Married -  Male Not married & 0.26 & 0.03 & 9.00 & 0.0000 \\ 
    Male Married -  Female Not married & 0.94 & 0.03 & 27.73 & 0.0000 \\ 
    Female Married -  Female Not married & 0.78 & 0.03 & 27.52 & 0.0000 \\ 
    Male Not married -  Female Not married & 0.67 & 0.03 & 22.30 & 0.0000 \\ 
    Female Married -  Male Not married & 0.11 & 0.04 & 3.03 & 0.0024 \\ 
   \hline
\end{tabular}
\caption{Post-hoc test of Gender and Marital status} 
\label{tab:gender:ms}
\end{table}

% latex table generated in R 3.3.1 by xtable 1.8-2 package
% Mon Mar 13 12:28:05 2017
\begin{table}[H]
\footnotesize
\centering
\begin{tabular}{lrrrr}
  \hline
Factor Levels & Estimate & Standard Error & t-value & p-value \\ 
  \hline
  Male High School or less -  Female High School or less & 0.28 & 0.03 & 7.98 & 0.0000 \\ 
    Male High School or less -  Male Some college, diploma, assoc & -0.55 & 0.05 & -12.14 & 0.0000 \\ 
    Male High School or less -  Male Bachelors or higher & -1.81 & 0.05 & -33.25 & 0.0000 \\ 
    Male High School or less -  Female Bachelors or higher & -1.28 & 0.05 & -23.46 & 0.0000 \\ 
    Female High School or less -  Male Some college, diploma, assoc & -0.82 & 0.05 & -17.49 & 0.0000 \\ 
    Female High School or less -  Female Some college, diploma, assoc & -0.39 & 0.04 & -9.01 & 0.0000 \\ 
    Female High School or less -  Male Bachelors or higher & -2.09 & 0.06 & -37.35 & 0.0000 \\ 
    Female High School or less -  Female Bachelors or higher & -1.56 & 0.05 & -29.82 & 0.0000 \\ 
    Male Some college, diploma, assoc -  Female Some college, diploma, assoc & 0.43 & 0.04 & 11.59 & 0.0000 \\ 
    Male Some college, diploma, assoc -  Male Bachelors or higher & -1.26 & 0.05 & -23.80 & 0.0000 \\ 
    Male Some college, diploma, assoc -  Female Bachelors or higher & -0.73 & 0.05 & -13.44 & 0.0000 \\ 
    Female Some college, diploma, assoc -  Male Bachelors or higher & -1.69 & 0.05 & -30.91 & 0.0000 \\ 
    Female Some college, diploma, assoc -  Female Bachelors or higher & -1.16 & 0.05 & -23.24 & 0.0000 \\ 
    Male Bachelors or higher -  Female Bachelors or higher & 0.53 & 0.04 & 12.35 & 0.0000 \\ 
    Male High School or less -  Female Some college, diploma, assoc & -0.12 & 0.05 & -2.56 & 0.0105 \\ 
   \hline
\end{tabular}
\caption{Post-hoc test of Gender and Education} 
\label{tab:gender_education}
\end{table}

% latex table generated in R 3.3.1 by xtable 1.8-2 package
% Mon Mar 13 12:35:52 2017
\begin{table}[H]
\footnotesize
\centering
\begin{tabular}{lrrrr}
  \hline
Factor Levels & Estimate & Standard Error & t-value & p-value \\ 
  \hline
  Married Black -  Not married Black & 0.54 & 0.05 & 11.34 & 0.0000 \\ 
    Married Black -  Not married Hispanic & 0.38 & 0.07 & 5.12 & 0.0000 \\ 
    Married Black -  Married Others & -0.55 & 0.08 & -6.83 & 0.0000 \\ 
    Married Black -  Married White & -1.05 & 0.06 & -18.44 & 0.0000 \\ 
    Married Black -  Not married White & -0.30 & 0.06 & -5.18 & 0.0000 \\ 
    Not married Black -  Married Hispanic & -0.39 & 0.07 & -5.70 & 0.0000 \\ 
    Not married Black -  Married Others & -1.10 & 0.08 & -14.39 & 0.0000 \\ 
    Not married Black -  Not married Others & -0.53 & 0.08 & -6.88 & 0.0000 \\ 
    Not married Black -  Married White & -1.59 & 0.05 & -32.10 & 0.0000 \\ 
    Not married Black -  Not married White & -0.84 & 0.05 & -16.91 & 0.0000 \\ 
    Married Hispanic -  Not married Hispanic & 0.22 & 0.04 & 5.25 & 0.0000 \\ 
    Married Hispanic -  Married Others & -0.71 & 0.08 & -8.99 & 0.0000 \\ 
    Married Hispanic -  Married White & -1.21 & 0.05 & -22.92 & 0.0000 \\ 
    Married Hispanic -  Not married White & -0.45 & 0.05 & -8.57 & 0.0000 \\ 
    Not married Hispanic -  Married Others & -0.93 & 0.08 & -11.72 & 0.0000 \\ 
    Not married Hispanic -  Not married Others & -0.36 & 0.08 & -4.55 & 0.0000 \\ 
    Not married Hispanic -  Married White & -1.43 & 0.05 & -26.45 & 0.0000 \\ 
    Not married Hispanic -  Not married White & -0.68 & 0.05 & -12.54 & 0.0000 \\ 
    Married Others -  Not married Others & 0.57 & 0.07 & 8.59 & 0.0000 \\ 
    Married Others -  Married White & -0.50 & 0.06 & -7.74 & 0.0000 \\ 
    Not married Others -  Married White & -1.07 & 0.06 & -16.50 & 0.0000 \\ 
    Not married Others -  Not married White & -0.31 & 0.06 & -4.84 & 0.0000 \\ 
    Married White -  Not married White & 0.75 & 0.02 & 40.59 & 0.0000 \\ 
    Married Others -  Not married White & 0.26 & 0.06 & 4.01 & 0.0001 \\ 
    Not married Black -  Not married Hispanic & -0.16 & 0.07 & -2.38 & 0.0193 \\ 
    Married Black -  Married Hispanic & 0.16 & 0.07 & 2.14 & 0.0347 \\ 
    Married Hispanic -  Not married Others & -0.14 & 0.08 & -1.76 & 0.0820 \\ 
    Married Black -  Not married Others & 0.02 & 0.08 & 0.21 & 0.8318 \\ 
   \hline
\end{tabular}
\caption{Post-hoc test of Marital status and Ethnicity} 
\label{tab:ms_race_origin}
\end{table}

% latex table generated in R 3.3.1 by xtable 1.8-2 package
% Mon Mar 13 12:40:24 2017
\begin{table}[H]
\footnotesize
\centering
\begin{tabular}{lrrrr}
  \hline
Factor Levels & Estimate & Standard Error & t-value & p-value \\ 
  \hline
  Married High School or less -  Not married High School or less & 0.41 & 0.03 & 13.63 & 0.0000 \\ 
    Married High School or less -  Married Some college, diploma, assoc & -0.53 & 0.04 & -12.40 & 0.0000 \\ 
    Married High School or less -  Married Bachelors or higher & -1.80 & 0.05 & -35.43 & 0.0000 \\ 
    Married High School or less -  Not married Bachelors or higher & -1.16 & 0.05 & -21.21 & 0.0000 \\ 
    Not married High School or less -  Married Some college, diploma, assoc & -0.93 & 0.04 & -20.91 & 0.0000 \\ 
    Not married High School or less -  Not married Some college, diploma, assoc & -0.42 & 0.04 & -10.19 & 0.0000 \\ 
    Not married High School or less -  Married Bachelors or higher & -2.21 & 0.05 & -42.18 & 0.0000 \\ 
    Not married High School or less -  Not married Bachelors or higher & -1.57 & 0.05 & -31.25 & 0.0000 \\ 
    Married Some college, diploma, assoc -  Not married Some college, diploma, assoc & 0.52 & 0.03 & 16.60 & 0.0000 \\ 
    Married Some college, diploma, assoc -  Married Bachelors or higher & -1.28 & 0.05 & -26.12 & 0.0000 \\ 
    Married Some college, diploma, assoc -  Not married Bachelors or higher & -0.63 & 0.05 & -11.94 & 0.0000 \\ 
    Not married Some college, diploma, assoc -  Married Bachelors or higher & -1.79 & 0.05 & -34.60 & 0.0000 \\ 
    Not married Some college, diploma, assoc -  Not married Bachelors or higher & -1.15 & 0.05 & -23.79 & 0.0000 \\ 
    Married Bachelors or higher -  Not married Bachelors or higher & 0.64 & 0.03 & 18.55 & 0.0000 \\ 
    Married High School or less -  Not married Some college, diploma, assoc & -0.01 & 0.05 & -0.18 & 0.8540 \\ 
   \hline
\end{tabular}
\caption{Post-hoc test of Marital status and Education} 
\label{tab:ms_education}
\end{table}


% latex table generated in R 3.3.1 by xtable 1.8-2 package
% Mon Mar 13 12:57:48 2017
\begin{table}[H]
\footnotesize
\centering
\begin{tabular}{lrrrr}
  \hline
Factor Levels & Estimate & Standard Error & t-value & p-value \\ 
  \hline
 Male - Female & -0.03 & 0.01 & -6.56 & 0.0000 \\ 
   \hline
\end{tabular}
\caption{Post-hoc test of Gender for households with Disability} 
\label{tab:genderDisb}
\end{table}

% latex table generated in R 3.3.1 by xtable 1.8-2 package
% Mon Mar 13 12:57:59 2017
\begin{table}[H]
\footnotesize
\centering
\begin{tabular}{lrrrr}
  \hline
Factor Levels & Estimate & Standard Error & t-value & p-value \\ 
  \hline
 Married - Not married & -0.10 & 0.01 & -17.80 & 0.0000 \\ 
   \hline
\end{tabular}
\caption{Post-hoc test of Marital status for households with Disability} 
\label{tab:msDisb}
\end{table}

% latex table generated in R 3.3.1 by xtable 1.8-2 package
% Mon Mar 13 12:58:26 2017
\begin{table}[H]
\footnotesize
\centering
\begin{tabular}{lrrrr}
  \hline
Factor Levels & Estimate & Standard Error & t-value & p-value \\ 
  \hline
 Black - White & 0.05 & 0.01 & 5.54 & 0.0000 \\ 
   Others - White & 0.07 & 0.01 & 6.17 & 0.0000 \\ 
   Hispanic - White & 0.04 & 0.01 & 3.92 & 0.0002 \\ 
   Hispanic - Others & -0.03 & 0.02 & -1.80 & 0.1077 \\ 
   Black - Others & -0.02 & 0.01 & -1.14 & 0.3030 \\ 
   Black - Hispanic & 0.01 & 0.01 & 0.80 & 0.4227 \\ 
   \hline
\end{tabular}
\caption{Post-hoc test of Ethnicity for households with Disability} 
\label{tab:race_originDisb}
\end{table}

% latex table generated in R 3.3.1 by xtable 1.8-2 package
% Mon Mar 13 12:58:43 2017
\begin{table}[H]
\footnotesize
\centering
\begin{tabular}{lrrrr}
  \hline
Factor Levels & Estimate & Standard Error & t-value & p-value \\ 
  \hline
 High School or less - Some college, diploma, assoc & 0.07 & 0.01 & 9.41 & 0.0000 \\ 
   High School or less - Bachelors or higher & 0.17 & 0.01 & 17.36 & 0.0000 \\ 
   Some college, diploma, assoc - Bachelors or higher & 0.10 & 0.01 & 10.33 & 0.0000 \\ 
   \hline
\end{tabular}
\caption{Post-hoc test of Education for households with Disability} 
\label{tab:educationDisb}
\end{table}

% latex table generated in R 3.3.1 by xtable 1.8-2 package
% Mon Mar 13 12:59:09 2017
\begin{table}[H]
\footnotesize
\centering
\begin{tabular}{lrrrr}
  \hline
Factor Levels & Estimate & Standard Error & t-value & p-value \\ 
  \hline
  Male Married -  Male Not married & -0.08 & 0.01 & -11.55 & 0.0000 \\ 
    Male Married -  Female Not married & -0.14 & 0.01 & -18.46 & 0.0000 \\ 
    Female Married -  Male Not married & -0.07 & 0.01 & -8.79 & 0.0000 \\ 
    Female Married -  Female Not married & -0.12 & 0.01 & -17.79 & 0.0000 \\ 
    Male Not married -  Female Not married & -0.05 & 0.01 & -7.96 & 0.0000 \\ 
    Male Married -  Female Married & -0.01 & 0.01 & -2.23 & 0.0255 \\ 
   \hline
\end{tabular}
\caption{Post-hoc test of Gender and Marital status for households with Disability} 
\label{tab:gender:msDisb}
\end{table}

% latex table generated in R 3.3.1 by xtable 1.8-2 package
% Mon Mar 13 13:00:38 2017
\begin{table}[H]
\footnotesize
\centering
\begin{tabular}{lrrrr}
  \hline
Factor Levels & Estimate & Standard Error & t-value & p-value \\ 
  \hline
  Married Black -  Not married Black & -0.17 & 0.01 & -14.94 & 0.0000 \\ 
    Married Black -  Not married Hispanic & -0.09 & 0.02 & -5.16 & 0.0000 \\ 
    Married Black -  Not married Others & -0.16 & 0.02 & -9.22 & 0.0000 \\ 
    Married Black -  Not married White & -0.07 & 0.01 & -5.60 & 0.0000 \\ 
    Not married Black -  Married Hispanic & 0.11 & 0.02 & 7.29 & 0.0000 \\ 
    Not married Black -  Not married Hispanic & 0.08 & 0.02 & 4.94 & 0.0000 \\ 
    Not married Black -  Married Others & 0.13 & 0.02 & 8.38 & 0.0000 \\ 
    Not married Black -  Married White & 0.18 & 0.01 & 16.62 & 0.0000 \\ 
    Not married Black -  Not married White & 0.10 & 0.01 & 8.84 & 0.0000 \\ 
    Married Hispanic -  Not married Others & -0.11 & 0.02 & -6.10 & 0.0000 \\ 
    Married Hispanic -  Married White & 0.07 & 0.01 & 5.72 & 0.0000 \\ 
    Not married Hispanic -  Not married Others & -0.08 & 0.02 & -4.17 & 0.0000 \\ 
    Not married Hispanic -  Married White & 0.10 & 0.01 & 7.79 & 0.0000 \\ 
    Married Others -  Not married Others & -0.13 & 0.02 & -8.57 & 0.0000 \\ 
    Not married Others -  Married White & 0.18 & 0.01 & 12.26 & 0.0000 \\ 
    Not married Others -  Not married White & 0.10 & 0.01 & 6.50 & 0.0000 \\ 
    Married White -  Not married White & -0.08 & 0.01 & -16.53 & 0.0000 \\ 
    Married Others -  Married White & 0.05 & 0.01 & 3.46 & 0.0008 \\ 
    Married Black -  Married Hispanic & -0.05 & 0.02 & -3.39 & 0.0010 \\ 
    Not married Hispanic -  Married Others & 0.06 & 0.02 & 3.13 & 0.0024 \\ 
    Married Hispanic -  Not married Hispanic & -0.03 & 0.01 & -3.01 & 0.0035 \\ 
    Married Others -  Not married White & -0.04 & 0.01 & -2.77 & 0.0073 \\ 
    Married Black -  Married Others & -0.03 & 0.02 & -1.92 & 0.0666 \\ 
    Not married Hispanic -  Not married White & 0.02 & 0.01 & 1.38 & 0.1941 \\ 
    Married Hispanic -  Married Others & 0.02 & 0.02 & 1.33 & 0.2070 \\ 
    Married Hispanic -  Not married White & -0.01 & 0.01 & -1.21 & 0.2420 \\ 
    Married Black -  Married White & 0.01 & 0.01 & 1.15 & 0.2606 \\ 
    Not married Black -  Not married Others & 0.00 & 0.02 & 0.02 & 0.9807 \\ 
   \hline
\end{tabular}
\caption{Post-hoc test of Marital status and Ethnicity for households with Disability} 
\label{tab:ms:race_originDisb}
\end{table}

% latex table generated in R 3.3.1 by xtable 1.8-2 package
% Mon Mar 13 13:01:30 2017
\begin{table}[H]
\footnotesize
\centering
\begin{tabular}{lrrrr}
  \hline
Factor Levels & Estimate & Standard Error & t-value & p-value \\ 
  \hline
  Married High School or less -  Not married High School or less & -0.13 & 0.01 & -18.85 & 0.0000 \\ 
    Married High School or less -  Married Some college, diploma, assoc & 0.06 & 0.01 & 6.80 & 0.0000 \\ 
    Married High School or less -  Not married Some college, diploma, assoc & -0.05 & 0.01 & -5.05 & 0.0000 \\ 
    Married High School or less -  Married Bachelors or higher & 0.14 & 0.01 & 12.63 & 0.0000 \\ 
    Married High School or less -  Not married Bachelors or higher & 0.07 & 0.01 & 6.05 & 0.0000 \\ 
    Not married High School or less -  Married Some college, diploma, assoc & 0.19 & 0.01 & 20.38 & 0.0000 \\ 
    Not married High School or less -  Not married Some college, diploma, assoc & 0.08 & 0.01 & 9.64 & 0.0000 \\ 
    Not married High School or less -  Married Bachelors or higher & 0.27 & 0.01 & 23.58 & 0.0000 \\ 
    Not married High School or less -  Not married Bachelors or higher & 0.21 & 0.01 & 18.17 & 0.0000 \\ 
    Married Some college, diploma, assoc -  Not married Some college, diploma, assoc & -0.11 & 0.01 & -14.49 & 0.0000 \\ 
    Married Some college, diploma, assoc -  Married Bachelors or higher & 0.08 & 0.01 & 7.16 & 0.0000 \\ 
    Not married Some college, diploma, assoc -  Married Bachelors or higher & 0.19 & 0.01 & 16.19 & 0.0000 \\ 
    Not married Some college, diploma, assoc -  Not married Bachelors or higher & 0.12 & 0.01 & 11.17 & 0.0000 \\ 
    Married Bachelors or higher -  Not married Bachelors or higher & -0.07 & 0.01 & -7.06 & 0.0000 \\ 
    Married Some college, diploma, assoc -  Not married Bachelors or higher & 0.01 & 0.01 & 1.10 & 0.2723 \\ 
   \hline
\end{tabular}
\caption{Post-hoc test of Marital status and Education for households with Disability} 
\label{tab:ms:educationDisb}
\end{table}




\newpage
\bibliography{bibTex_Reference}
\end{document}
