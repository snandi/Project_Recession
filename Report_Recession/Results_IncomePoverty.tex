\documentclass[11pt]{extarticle} %extarticle for fontsizes other than 10, 11 And 12
%\documentclass[11p]{article}

%%%%%%%%%%%%%%%%%%%%%%%%%%%%%%%%%%%%%%%%%%%%%%%%%%%%%%%%%%%%%%%%%%%%%%
%% Input header file 
%%%%%%%%%%%%%%%%%%%%%%%%%%%%%%%%%%%%%%%%%%%%%%%%%%%%%%%%%%%%%%%%%%%%%%
%%%%%%%%%%%%%%%%%%%%%%%% Packages %%%%%%%%%%%%%%%%%%%%%%%%
\usepackage{amscd}
\usepackage{amsmath}
\usepackage{amssymb}
\usepackage{amsthm}
\usepackage{amsxtra}
\usepackage{animate}
\usepackage{bbold}
%\usepackage{bigints}
\usepackage{caption}    %% For multiple line captions
\usepackage{color, colortbl}
\usepackage{dsfont}
\usepackage{enumerate}
\usepackage[mathscr]{eucal}
%\usepackage{fancyhdr}
\usepackage{float}
%\usepackage{fullpage}  %% Dont use this for beamer presentations
\usepackage{geometry}
\usepackage{graphicx}
\usepackage{hyperref}
\usepackage{indentfirst}
\usepackage{latexsym}
\usepackage{listings}
\usepackage{longtable}  %% to add pagebreaks in between table
\usepackage{lscape}
\usepackage{mathtools}
\usepackage{microtype}
\usepackage{multirow}
\usepackage{natbib}
\usepackage{pdfpages}
\usepackage{setspace}   %% Allows to set double or single space
\usepackage{tcolorbox}  %% For colored textboxes
\usepackage{verbatim}
\usepackage{wrapfig}
\usepackage{xargs}
\usepackage{xcolor}
\DeclareGraphicsExtensions{.pdf,.png,.jpg, .jpeg}
\definecolor{LightCyan}{rgb}{0.88,1,1}

\usepackage{array}
\newcolumntype{C}[1]{>{\centering\arraybackslash}p{#1}}  %% For wrapping text in table headers

%%%%%%%%%%%%%%%%%%%%%%%% Commands %%%%%%%%%%%%%%%%%%%%%%%%
\newcommand{\Sup}{\textsuperscript}
\newcommand{\Exp}{\mathds{E}}
\newcommand{\Prob}{\mathds{P}}
\newcommand{\Z}{\mathds{Z}}
\newcommand{\Ind}{\mathds{1}}
\newcommand{\A}{\mathcal{A}}
\newcommand{\F}{\mathcal{F}}
%\newcommand{\G}{\mathcal{G}}
\newcommand{\I}{\mathcal{I}}
\newcommand{\R}{\mathcal{R}}
\newcommand{\Y}{\mathcal{Y}}
\newcommand{\Real}{\mathbb{R}}
\newcommand{\be}{\begin{equation}}
\newcommand{\ee}{\end{equation}}
\newcommand{\bes}{\begin{equation*}}
\newcommand{\ees}{\end{equation*}}
\newcommand{\union}{\bigcup}
\newcommand{\intersect}{\bigcap}
\newcommand{\Ybar}{\overline{Y}}
\newcommand{\ybar}{\bar{y}}
\newcommand{\Xbar}{\overline{X}}
\newcommand{\xbar}{\bar{x}}
\newcommand{\betahat}{\hat{\beta}}
\newcommand{\Yhat}{\widehat{Y}}
\newcommand{\yhat}{\hat{y}}
\newcommand{\Xhat}{\widehat{X}}
\newcommand{\xhat}{\hat{x}}
\newcommand{\E}[1]{\operatorname{E}\left[ #1 \right]}
%\newcommand{\Var}[1]{\operatorname{Var}\left( #1 \right)}
\newcommand{\Var}{\operatorname{Var}}
\newcommand{\Cov}[2]{\operatorname{Cov}\left( #1,#2 \right)}
\newcommand{\N}[2][1=\mu, 2=\sigma^2]{\operatorname{N}\left( #1,#2 \right)}
\newcommand{\bp}[1]{\left( #1 \right)}
\newcommand{\bsb}[1]{\left[ #1 \right]}
\newcommand{\bcb}[1]{\left\{ #1 \right\}}
\newcommand*{\permcomb}[4][0mu]{{{}^{#3}\mkern#1#2_{#4}}}
\newcommand*{\perm}[1][-3mu]{\permcomb[#1]{P}}
\newcommand*{\comb}[1][-1mu]{\permcomb[#1]{C}}
\newcommand{\indep}{\rotatebox[origin=c]{90}{$\models$}}

\DeclareMathOperator*{\argmin}{arg\,min}


\begin{document}
\doublespacing
%\SweaveOpts{concordance=TRUE}
\bibliographystyle{plain}  %Choose a bibliograhpic style

\title{Effects of Great Recession on Income Poverty}
\author{Subharati Ghosh, Subhrangshu Nandi, Susan Murphy \\
%  Statistics PhD Student, \\
%  Research Assistant,
%  Laboratory of Molecular and Computational Genomics, \\
%  University of Wisconsin - Madison}
%\date{February 16, 2015}
\date{}
}

\maketitle

\section{Study Aim}
The aim of this study is to analyze how households with a working age adult with disability compare with households with no adult with disability, during the {\emph{great recession}}, using ``Income Poverty'' as a measure of economic wellbeing, when controlling for demographic factors such as gender, marital status, education, race and origin. 

\section{Sample}
For this analysis data from US Census Bureau's SIPP 2008 panel survey was used. {\footnote{For more information on the SIPP 2008 panel schedule, please refer to this \href{http://www.census.gov/programs-surveys/sipp/data/2008-panel.html}{US Census Bureau website}}}. Questions on whether the households had a working age adult with disability were asked in wave-6 of the survey, which ended in August, 2010. Households that participated in wave-6 were included in our sample. There were a total of 34,850 households in wave six. Survey data upto wave-15 were used in our sample. Survey results from July, 2008 through June, 2013 were included in the analysis. Households whose reference person remained the same throughout the 2008 panel were kept in the sample. The reference persons of the households were also required to be 18 years or older throughout the 2008 panel. The final sample had 33,547 households that satisfied all the inclusion criteria. 

\section{Methods}
Total monthly household income was divided by monthly federal poverty level (FPL) and then averaged over quarters to estimate FPL100-ratio. An FPL100-ratio lower than one in any quarter indicated the household was below 100\% Federal poverty level in that quarter. Averaging monthly values over a quarter reduced the noise in the response variable by eliminating the month-by-month variability in the income data. In the sample, the monthly income data ranged from -\$27,180 to \$108,900, the average being \$5240 and median \$3,874. The negative incomes were associated with households owning business that incurred lossed in those months. The FPL100-ratio ranged from -17.95 to 89.48, with the average being 3.817 and the median 2.924. In the sample, 7,865 out of 33,547 households (23.44\%) had at least one working age adult with disability during the observed time period, as identified in wave-6 of the survey. Data from July 2008 (2008-Q3) through May 2013 (2013-Q1) were analyzed. This period overlapped with twelve of the eighteen months {\footnote{\href{http://www.nber.org/cycles/}{NBER Recession Cycles}}} of the ``Great recession'' and its long wake. 

\subsection{Mixed-effects model for income poverty}
A mixed (fixed and random) effects model was fit to estimate the impact of the presence of a working-age adult with disability in a household on its FPL100-ratio. Let $Y$ denote the vector or responses (FPL100-ratio). Let $\theta$ denote the vector of fixed effect factors like gender, marital status, education level, race, origin of household head, along with their interactions. Let $\beta$ denote another fixed effect of time, represented as quarters, starting from 2008-Q3 and ending in 2013-Q1. Let $b$ denote the household level random effect (random intercept). Then, the mixed-effects (\cite{Fitzmaurice_2012_Applied}) model for the responses, for each household $i$ can be written as
\begin{equation}
Y_{ij} = X_i\theta + \beta t_j + b_i + \epsilon_{ij}
\label{eq:MixedEffects1}
\end{equation}
where, $\epsilon_{ij}$ are regarded as measurement errors, $i$ goes from $1$ to $H$, the number of households, $j$ goes from $1$ to $T$, the total number of quarterly observations for every household. In this model, the response from the $i^{th}$ household at time $t_j$ is assumed to differ from the population mean $X_i\theta + \beta t_j$ by a household effect $b_i$ and a within household measurement error $\epsilon_{ij}$. The within household and between household errors are assumed to be normal and independent ($b_i \sim \N(0, \sigma^2_b),\ \ \epsilon_{ij} \sim \N(0, \sigma^2), \ \ b_i \indep \epsilon_{ij}, \forall i, j$). The effect of ``time'' is a fixed effect and it could be considered part of the fixed effect design matrix $X$. However, the problem of interest is to test a linear hypothesis about the disability-by-time interaction, to detect the effect of disability on the mean response over the period of the study. Hence, the ``time'' covariate is denoted separately. During estimation it will be estimated as a fixed effect. 

\noindent
{\bf{Demographic factors}}\\
The demographic factors were considered as fixed effects. The factors included in this analysis were {\emph{gender}}, {\emph{marital status}}, {\emph{education}}, and {\emph{race/origin}} of household head. The {\emph{Race/Origin}} factor included ``non-hispanic white'', ``non-hispanic black'', ``hispanic'' and ``others''. For simplicity, ``white'' and ``black'' would indicate categories ``non-hispanic white'' and ``non-hispanic black''. Gender of household head had two categories: ``male'' and ``female''. Education of household head has three categories: ``high-school or less'', ``some college, diploma, associated degrees'' and ``bachelors or higher''. Marital status of household heads has two categories: ``married'' and ``not married''. Divorced or widowed household heads were considered in the ``not married'' category. 

\noindent
{\bf{Computational software}}\\
All analysis were conducted using the statistical software R (\cite{R}), version 3.3.1. The mixed effects models were fit using the R-package ``lme4'' (\cite{R-lme4}) and all hypothesis tests were done using the R package ``lmerTest'' (\cite{Kuznetsova_etal_2015_R-lmerTest}). The final model was fit with some of the fixed effect factors along with their interactions after performing ``backward elimination'' on the full model. Elimination of the fixed effects were done by the principle of marginality, that is: the highest order interactions are tested first: if they are significant, the lower order effects were included in the model without testing for significance. The p-values for the fixed effects are estimated from the F statistics, with ``Satterthwaite'' approximation (\cite{Satterthwaite_1946_Biometrics}) denominator degrees of freedom. The p-values for the random effect were computed from likelihood ratio tests (\cite{Morrell_1998_Biometrics}). 

\subsection{Model for income poverty, with baseline correction} \label{baselineCorrection}
In order to isolate the effect of the {\emph{great recession}} on income poverty (response), measured by FPL100-ratio, the values of the response at the beginning of study period (2008-Q3) were subtracted from each household's responses. Consequently, all FPL100-ratios of all households at 2008-Q3 were zero. The same model as in eq \ref{eq:MixedEffects1} was fit to this baseline-corrected responses. The problem of interest was to test a linear hypothesis about the disability-by-time interaction. 

\subsection{Post-hoc tests}
Post-hoc tests were conducted between categories of all demographic factors and their interactions, by calculating differences of ``Least Squares Means'' using R package ``lmerTest'' (\cite{Kuznetsova_etal_2015_R-lmerTest}), with ``Satterthwaite'' approximation (\cite{Satterthwaite_1946_Biometrics}) of the denominator degrees of freedom. 

\section{Results}
Table \ref{tab:FixedEffectsBetas} shows the coefficients of time (measured in year-quarters), and disability and the interaction between them. The first model includes the baseline FPL100-ratio and the second model does not. The rationale for the second model is explained in section \ref{baselineCorrection}. Both these models include the demographic factors. ANOVA tables of these demographic factors are in tables \ref{tab:Anova1} and \ref{tab:Anova2} respectively, for the two models. The coefficients in the two models are comparable. For example, $\beta_{\text{time}} = -0.0515, p < 0.01$ in model 1, indicating, FPL100-ratio decreased by $0.05$ every quarter, during the study period. The coefficient of {\emph{Disability}} is $\beta_D = -30.8394, p < 0.01$ in model 1 indicate that households with a working age adult with disability had their FPL100-ratios $30.84$ lower, on an average, compared to households without any working age adult with disability. Next, we observe that the coefficient of interaction between time and disability ($\beta_{t*D} = 0.0151, p < 0.05$) is positive. This implies that the slope of FPL100-ratio with time for households with disability is $-0.0364 (-0.0515 + 0.0151)$, which is less negative than the households without disability. This apparent contradictory finding leads us to conclude that households with disability although had significantly worse FPL100-ratio throughout the study period, the households without disability experienced more severe declines in their FPL100-ratios. This could throw some light on the impact of different supplementary coverage programs on households with disability. 

Although the coefficients of the two different models (without and with baseline correction) were similar, we feel Model 2 is more appropriate for reasons explained in \ref{baselineCorrection}. For all the demographic factors and for interactions between them, post-hoc tests were conducted to identify the pairwise differences in mean FPL100-ratios of different subgroups. 

\noindent
{\bf{Race/Origin}} \\
In table \ref{tab:RaceOrigin} we can see that regardless of disability, in Model1, households with a ``black'' ($\mu_{\text{diff}} = -0.53, p = 0.0000$) or ``hispanic'' ($\mu_{\text{diff}} = -0.46, p = 0.0000$) race/origin as household head is worse off compared to those with ``others'' race/origin. The discrepancy is higher between ``black'' and ``white'' household heads ($\mu_{\text{diff}} = -0.90, p = 0.0000$). When corrected for baseline, the relationships reverse in sign, indicating that during the study period, ``black'' and ``hispanic'' households fared marginally better than ``white'' and ``others''. When considering only the households with disability (table \ref{tab:DisabRaceOrigin}), there is no significant difference between the different races. \\
\noindent
{\bf{Education}} \\
In table \ref{tab:Education}, we see that regardless of disability, in Model1, households in ``high school or less'' is worse off than ``some college, diploma'' ($\mu_{\text{diff}} = -0.47, p = 0.0000$), which is in turn worse off than ``bachelors or higher'' ($\mu_{\text{diff}} = -1.29, p = 0.0000$). In Model 2, only the difference between ``some college, diploma'' and ``bachelors or higher'' groups are significant ($\mu_{\text{diff}} = -0.08, p = 0.014$. When considering only the households with disability (table \ref{tab:DisabEducation}), there is no significant difference between the different education groups. \\
\noindent
{\bf{Interaction of Gender and Marital status}} \\
In table \ref{tab:GenderMS}, we see that in Model 1 ``female, not married'' households are worse off than ``male, married'' ($\mu_{\text{diff}} = -0.92, p = 0.0000$), than ``female, married'' ($\mu_{\text{diff}} = -0.75, p = 0.0000$) and ``male, not married'' ($\mu_{\text{diff}} = -0.65, p = 0.0000$). In Model 2, similar relationships hold, with smaller mean difference. Even when considering only the households with disability (table \ref{tab:DisabGenderMS}), the same relationships hold between the groups. We conclude that ``female, not married'' households fared the most poorly through our study period. \\
\noindent
{\bf{Interaction of Gender and Race/Origin}} \\
Focusing on the households In table \ref{tab:GenderRaceOrigin}, we see that in Model 1,      \\
\noindent
{\bf{Interaction of Race/Origin and Marital status}} \\
In table


\section{Limitations}
\begin{enumerate}
\item Although a linear mixed effects regression model discovered some conventional and some interesting patterns in the relationships between response and demographic factors, along with disability, the trajectory of income poverty over the study period for some households were not linear. This modeling approach does not capture trajectory shapes of individual households. A non-parametric fitting of the income poverty trajectories could be tried as a pre-processing step before testing for differences in behavior between different groups of households. 
\end{enumerate}

\newpage
\section{Tables}

\noindent
\begin{table}[H] 
\centering 
\footnotesize
%\begin{tabular}{@{\extracolsep{5pt}}lccc|c}
\begin{tabular}{l|c|c}
\hline 
\hline 
% & \multicolumn{4}{c}{\textit{Dependent variable:}} \\ 
& Model1 (with Baseline) & Model2 (with No Baseline) \\
\hline 
Year-Quarter (time)	&	$-0.0515^{***}$		&	$-0.0527^{***}$ 	\\
			&	$(0.0021)$		&	$(0.0021)$		\\
Disability		&	$-30.8394^{***}$	&	$-29.6874^{***}$	\\    
			&	$(8.7393)$		&	$(8.7195)$		\\
Year-Quarter*Disability	&	$0.0151^{**}$		&	$0.0147^{**}$		\\
			&	$(0.0043)$		&	$(0.0043)$		\\
\hline 
\hline 
\multicolumn{3}{r}{\textit{Note:}  $^{*}$p$<$0.1; $^{**}$p$<$0.05; $^{***}$p$<$0.01} \\ 

\end{tabular}
\caption{Coefficients of Fixed effects in regression models} 
\label{tab:FixedEffectsBetas} 
\end{table}

\noindent
% latex table generated in R 3.3.1 by xtable 1.8-2 package
% Wed Mar  8 11:16:09 2017
\begin{table}[H]
\footnotesize
\centering
\begin{tabular}{rrrrrr}
  \hline
  Factor & Sum Sq & Mean Sq & NumDF & F.value & p.value \\ 
  \hline
  race\_origin & 5752.68 & 2876.34 & 3 & 238.71 & 0.0000 \\ 
  education & 29404.45 & 14702.22 & 2 & 1220.14 & 0.0000 \\ 
  gender:ms & 1813.16 & 1813.16 & 1 & 150.47 & 0.0000 \\ 
  ms:race\_origin & 2070.73 & 690.24 & 3 & 57.28 & 0.0000 \\ 
  gender:adult\_disb & 237.05 & 237.05 & 1 & 19.67 & 0.0000 \\ 
  yearqtrNum & 145.47 & 145.47 & 1 & 12.07 & 0.0005 \\ 
  yearqtrNum:adult\_disb & 145.47 & 145.47 & 1 & 12.07 & 0.0005 \\ 
  adult\_disb:education & 109.34 & 54.67 & 2 & 4.54 & 0.0107 \\ 
  gender:ms:adult\_disb & 56.78 & 56.78 & 1 & 4.71 & 0.0299 \\ 
  gender:race\_origin & 104.06 & 34.69 & 3 & 2.88 & 0.0345 \\ 
  race\_origin:adult\_disb & 26.90 & 8.97 & 3 & 0.74 & 0.5255 \\ 
  ms:adult\_disb & 0.12 & 0.12 & 1 & 0.01 & 0.9222 \\ 
  \hline
\end{tabular}
\caption{Model 1: FPL100 vs demographic factors, time and disability} 
\label{tab:Anova1}
\end{table}

\noindent
% latex table generated in R 3.3.1 by xtable 1.8-2 package
% Wed Mar  8 11:33:06 2017
\begin{table}[H]
\footnotesize
\centering
\begin{tabular}{rrrrrr}
  \hline
  Factor & Sum Sq & Mean Sq & NumDF & F.value & p.value \\ 
  \hline
  gender:ms & 721.70 & 721.70 & 1 & 60.04 & 0.0000 \\ 
  ms:race\_origin & 556.33 & 185.44 & 3 & 15.43 & 0.0000 \\ 
  yearqtrNum & 138.47 & 138.47 & 1 & 11.52 & 0.0007 \\ 
  yearqtrNum:adult\_disb & 138.47 & 138.47 & 1 & 11.52 & 0.0007 \\ 
  gender:race\_origin & 181.68 & 60.56 & 3 & 5.04 & 0.0017 \\ 
  race\_origin & 151.47 & 75.73 & 3 & 6.30 & 0.0018 \\ 
  education & 72.76 & 36.38 & 2 & 3.03 & 0.0485 \\ 
  ms:adult\_disb & 31.83 & 31.83 & 1 & 2.65 & 0.1037 \\ 
  gender:adult\_disb & 19.75 & 19.75 & 1 & 1.64 & 0.1999 \\ 
  adult\_disb:education & 35.49 & 17.75 & 2 & 1.48 & 0.2285 \\ 
  gender:ms:adult\_disb & 14.38 & 14.38 & 1 & 1.20 & 0.2740 \\ 
  race\_origin:adult\_disb & 21.56 & 7.19 & 3 & 0.60 & 0.6163 \\ 
  \hline
\end{tabular}
\caption{Model 2: FPL100 vs demographic factors, time and disability with baseline differences in FPL100 eliminated} 
\label{tab:Anova2}
\end{table}

% latex table generated in R 3.3.1 by xtable 1.8-2 package
% Wed Mar  8 11:35:11 2017
\begin{table}[H]
\footnotesize
\centering
\begin{tabular}{rrrrrr}
  \hline
  Factor & Sum Sq & Mean Sq & NumDF & F.value & p.value \\ 
  \hline
  yearqtrNum & 1190.80 & 1190.80 & 1 & 132.39 & 0.0000 \\ 
  ms & 218.99 & 218.99 & 1 & 24.35 & 0.0000 \\ 
  ms:race\_origin & 279.40 & 93.13 & 3 & 10.35 & 0.0000 \\ 
  gender:ms & 161.88 & 161.88 & 1 & 18.00 & 0.0000 \\ 
  gender & 38.03 & 38.03 & 1 & 4.23 & 0.0398 \\ 
  race\_origin & 60.03 & 20.01 & 3 & 2.22 & 0.0831 \\ 
  gender:race\_origin & 42.92 & 14.31 & 3 & 1.59 & 0.1893 \\ 
  education & 19.70 & 9.85 & 2 & 1.10 & 0.3345 \\ 
  \hline
\end{tabular}
\caption{Model 3: FPL100 vs demographic factors, time and disability, disability only} 
\label{tab:Anova3}
\end{table}

% latex table generated in R 3.3.1 by xtable 1.8-2 package
% Wed Mar  8 11:41:59 2017
\begin{table}[H]
\footnotesize
\centering
\begin{tabular}{lrrrrrr}
  \hline
  Factor categories & Est 1 & Std Err 1 & p-value 1 & Est 2 & Std Err 2 & p-value 2 \\ 
  \hline
   Black - Others & -0.53 & 0.07 & 0.0000 & 0.22 & 0.07 & 0.0007 \\ 
   Black - White & -0.90 & 0.05 & 0.0000 & 0.14 & 0.04 & 0.0022 \\ 
   Hispanic - Others & -0.46 & 0.07 & 0.0000 & 0.16 & 0.07 & 0.0184 \\ 
   Hispanic - White & -0.83 & 0.05 & 0.0000 & 0.07 & 0.05 & 0.1144 \\ 
   Others - White & -0.37 & 0.06 & 0.0000 & -0.08 & 0.05 & 0.1159 \\ 
   Black - Hispanic & -0.07 & 0.07 & 0.2663 & 0.06 & 0.06 & 0.2924 \\ 
  \hline
\end{tabular}
\caption{Post-hoc test of Race/Origin} 
\label{tab:RaceOrigin}
\end{table}

% latex table generated in R 3.3.1 by xtable 1.8-2 package
% Wed Mar  8 11:46:27 2017
\begin{table}[H]
\footnotesize
\centering
\begin{tabular}{lrrrrrr}
  \hline
  Factor categories & Est 1 & Std Err 1 & p-value 1 & Est 2 & Std Err 2 & p-value 2 \\ 
  \hline
   Some college, diploma, assoc - Bachelors or higher & -1.29 & 0.03 & 0.0000 & -0.08 & 0.03 & 0.0140 \\ 
   High School or less - Bachelors or higher & -1.76 & 0.04 & 0.0000 & -0.06 & 0.03 & 0.0849 \\ 
   High School or less - Some college, diploma, assoc & -0.47 & 0.03 & 0.0000 & 0.02 & 0.03 & 0.4373 \\ 
  \hline
\end{tabular}
\caption{Post-hoc test of education} 
\label{tab:Education}
\end{table}

% latex table generated in R 3.3.1 by xtable 1.8-2 package
% Wed Mar  8 11:53:10 2017
\begin{table}[H]
\footnotesize
\centering
\begin{tabular}{lrrrrrr}
  \hline
  Factor categories & Est 1 & Std Err 1 & p-value 1 & Est 2 & Std Err 2 & p-value 2 \\ 
  \hline
    Female Married -  Female Not married & 0.74 & 0.03 & 0.0000 & 0.36 & 0.03 & 0.0000 \\ 
    Male Married -  Female Not married & 0.92 & 0.04 & 0.0000 & 0.37 & 0.04 & 0.0000 \\ 
    Male Not married -  Female Not married & 0.65 & 0.04 & 0.0000 & 0.29 & 0.04 & 0.0000 \\ 
    Male Married -  Male Not married & 0.27 & 0.03 & 0.0000 & 0.08 & 0.03 & 0.0110 \\ 
    Female Married -  Male Not married & 0.09 & 0.04 & 0.0436 & 0.07 & 0.04 & 0.0950 \\ 
    Male Married -  Female Married & 0.18 & 0.04 & 0.0000 & 0.01 & 0.04 & 0.8001 \\ 
  \hline
\end{tabular}
\caption{Post-hoc test of gender and marital status} 
\label{tab:GenderMS}
\end{table}

% latex table generated in R 3.3.1 by xtable 1.8-2 package
% Wed Mar  8 12:16:16 2017
\begin{table}[H]
\footnotesize
\centering
\begin{tabular}{lrrrrrr}
  \hline
  Factor categories & Est 1 & Std Err 1 & p-value 1 & Est 2 & Std Err 2 & p-value 2 \\ 
  \hline
    Married White 	-  Not married White & 0.75 & 0.02 & 0.0000 & 0.37 & 0.02 & 0.0000 \\ 
    Married Black 	-  Not married White & -0.27 & 0.06 & 0.0000 & 0.44 & 0.06 & 0.0000 \\ 
    Married Hispanic 	-  Not married White & -0.37 & 0.06 & 0.0000 & 0.30 & 0.05 & 0.0000 \\ 
    Married Black 	-  Not married Black & 0.52 & 0.05 & 0.0000 & 0.24 & 0.05 & 0.0000 \\ 
    Not married Black 	-  Not married White & -0.78 & 0.05 & 0.0000 & 0.20 & 0.05 & 0.0000 \\ 
    Not married Others 	-  Married White & -1.03 & 0.07 & 0.0000 & -0.36 & 0.06 & 0.0000 \\ 
    Married Black 	-  Not married Others & 0.01 & 0.09 & 0.8911 & 0.43 & 0.08 & 0.0000 \\ 
    Not married Hispanic -  Not married White & -0.54 & 0.06 & 0.0000 & 0.21 & 0.05 & 0.0001 \\ 
    Married Hispanic 	-  Not married Others & -0.09 & 0.09 & 0.3030 & 0.29 & 0.08 & 0.0001 \\ 
    Not married Black 	-  Married White & -1.54 & 0.05 & 0.0000 & -0.17 & 0.05 & 0.0004 \\ 
    Married Black 	-  Married Others & -0.56 & 0.09 & 0.0000 & 0.25 & 0.08 & 0.0013 \\ 
    Married Black 	-  Not married Hispanic & 0.27 & 0.08 & 0.0006 & 0.23 & 0.07 & 0.0013 \\ 
    Married Others 	-  Not married White & 0.29 & 0.07 & 0.0000 & 0.19 & 0.06 & 0.0026 \\ 
    Not married Hispanic -  Married White & -1.29 & 0.06 & 0.0000 & -0.16 & 0.05 & 0.0029 \\ 
    Married Others 	-  Married White & -0.46 & 0.07 & 0.0000 & -0.18 & 0.06 & 0.0041 \\ 
    Married Others 	-  Not married Others & 0.57 & 0.07 & 0.0000 & 0.18 & 0.06 & 0.0048 \\ 
    Not married Hispanic -  Not married Others & -0.26 & 0.09 & 0.0026 & 0.20 & 0.08 & 0.0095 \\ 
    Not married Black 	-  Not married Others & -0.51 & 0.08 & 0.0000 & 0.19 & 0.07 & 0.0103 \\ 
    Married Hispanic 	-  Not married Hispanic & 0.17 & 0.04 & 0.0001 & 0.09 & 0.04 & 0.0271 \\ 
    Married Black 	-  Married Hispanic & 0.10 & 0.08 & 0.1926 & 0.14 & 0.07 & 0.0476 \\ 
    Not married Black 	-  Married Hispanic & -0.42 & 0.07 & 0.0000 & -0.10 & 0.06 & 0.1051 \\ 
    Married Hispanic 	-  Married Others & -0.66 & 0.08 & 0.0000 & 0.11 & 0.08 & 0.1352 \\ 
    Married Hispanic 	-  Married White & -1.12 & 0.06 & 0.0000 & -0.07 & 0.05 & 0.1894 \\ 
    Married Black 	-  Married White & -1.02 & 0.06 & 0.0000 & 0.07 & 0.05 & 0.1899 \\ 
    Not married Hispanic -  Married Others & -0.83 & 0.09 & 0.0000 & 0.02 & 0.08 & 0.7908 \\ 
    Not married Black 	-  Not married Hispanic & -0.25 & 0.07 & 0.0006 & -0.01 & 0.07 & 0.8662 \\ 
    Not married Others 	-  Not married White & -0.28 & 0.07 & 0.0001 & 0.01 & 0.06 & 0.8850 \\ 
    Not married Black 	-  Married Others & -1.07 & 0.08 & 0.0000 & 0.01 & 0.07 & 0.8958 \\ 
  \hline
\end{tabular}
\caption{Post-hoc test of marital status and race/origin} 
\label{tab:RaceOriginMS}
\end{table}

% latex table generated in R 3.3.1 by xtable 1.8-2 package
% Wed Mar  8 12:39:19 2017
\begin{table}[H]
\footnotesize
\centering
\begin{tabular}{lrrrrrr}
  \hline
  Factor categories & Est 1 & Std Err 1 & p-value 1 & Est 2 & Std Err 2 & p-value 2 \\ 
  \hline
    Male White -  Female White & 0.40 & 0.03 & 0.0000 & 0.14 & 0.03 & 0.0000 \\ 
    Female Black -  Female Others & -0.38 & 0.09 & 0.0000 & 0.41 & 0.08 & 0.0000 \\ 
    Female Black -  Female White & -0.85 & 0.06 & 0.0000 & 0.22 & 0.05 & 0.0000 \\ 
    Male Others -  Female Others & 0.60 & 0.09 & 0.0000 & 0.35 & 0.08 & 0.0000 \\ 
    Female Others -  Male White & -0.87 & 0.08 & 0.0000 & -0.33 & 0.07 & 0.0000 \\ 
    Male Black -  Female Others & -0.08 & 0.09 & 0.3988 & 0.38 & 0.08 & 0.0000 \\ 
    Male Hispanic -  Female Others & 0.02 & 0.09 & 0.8198 & 0.40 & 0.08 & 0.0000 \\ 
    Male Hispanic -  Female White & -0.45 & 0.06 & 0.0000 & 0.21 & 0.06 & 0.0003 \\ 
    Male Black -  Female White & -0.56 & 0.07 & 0.0000 & 0.19 & 0.06 & 0.0012 \\ 
    Female Hispanic -  Female Others & -0.33 & 0.09 & 0.0003 & 0.26 & 0.08 & 0.0013 \\ 
    Female Others -  Female White & -0.48 & 0.08 & 0.0000 & -0.19 & 0.07 & 0.0055 \\ 
    Male Hispanic -  Female Hispanic & 0.35 & 0.06 & 0.0000 & 0.14 & 0.06 & 0.0162 \\ 
    Male Others -  Female White & 0.13 & 0.08 & 0.0901 & 0.16 & 0.07 & 0.0190 \\ 
    Female Black -  Female Hispanic & -0.05 & 0.08 & 0.5194 & 0.15 & 0.07 & 0.0323 \\ 
    Male Black -  Female Hispanic & 0.25 & 0.08 & 0.0027 & 0.12 & 0.07 & 0.1164 \\ 
    Female Black -  Male White & -1.25 & 0.06 & 0.0000 & 0.08 & 0.05 & 0.1262 \\ 
    Female Hispanic -  Female White & -0.81 & 0.06 & 0.0000 & 0.07 & 0.05 & 0.1650 \\ 
    Male Hispanic -  Male White & -0.85 & 0.06 & 0.0000 & 0.07 & 0.06 & 0.2074 \\ 
    Female Hispanic -  Male White & -1.20 & 0.06 & 0.0000 & -0.07 & 0.06 & 0.2291 \\ 
    Female Hispanic -  Male Others & -0.93 & 0.09 & 0.0000 & -0.09 & 0.08 & 0.2903 \\ 
    Male Black -  Male White & -0.95 & 0.07 & 0.0000 & 0.05 & 0.06 & 0.3831 \\ 
    Female Black -  Male Others & -0.98 & 0.09 & 0.0000 & 0.06 & 0.08 & 0.4639 \\ 
    Male Hispanic -  Male Others & -0.58 & 0.09 & 0.0000 & 0.05 & 0.08 & 0.5439 \\ 
    Male Black -  Female Black & 0.30 & 0.06 & 0.0000 & -0.03 & 0.06 & 0.6399 \\ 
    Male Black -  Male Others & -0.68 & 0.09 & 0.0000 & 0.03 & 0.08 & 0.7174 \\ 
    Male Others -  Male White & -0.27 & 0.08 & 0.0004 & 0.02 & 0.07 & 0.7651 \\ 
    Male Black -  Male Hispanic & -0.10 & 0.08 & 0.2292 & -0.02 & 0.08 & 0.7893 \\ 
    Female Black -  Male Hispanic & -0.40 & 0.08 & 0.0000 & 0.01 & 0.07 & 0.9142 \\ 
   \hline
\end{tabular}
\caption{Post-hoc test of gender with race/origin} 
\label{tab:GenderRaceOrigin}
\end{table}

{\bf{The following tables are only for households with a working-age adult with disability.}} \\

% latex table generated in R 3.3.1 by xtable 1.8-2 package
% Wed Mar  8 13:32:09 2017
\begin{table}[H]
\footnotesize
\centering
\begin{tabular}{lrrrr}
  \hline
  Factor categories & Estimate & Standard Error & t-value & p-value \\ 
  \hline
   Black - White & 0.14 & 0.07 & 2.09 & 0.0366 \\ 
   Black - Others & 0.17 & 0.10 & 1.74 & 0.0825 \\ 
   Hispanic - White & 0.11 & 0.07 & 1.56 & 0.1199 \\ 
   Hispanic - Others & 0.14 & 0.10 & 1.38 & 0.1681 \\ 
   Others - White & -0.03 & 0.08 & -0.36 & 0.7214 \\ 
   Black - Hispanic & 0.03 & 0.09 & 0.32 & 0.7471 \\ 
  \hline
\end{tabular}
\caption{Post-hoc test of race/origin} 
\label{tab:DisabRaceOrigin}
\end{table}

% latex table generated in R 3.3.1 by xtable 1.8-2 package
% Wed Mar  8 13:32:32 2017
\begin{table}[H]
\footnotesize
\centering
\begin{tabular}{lrrrr}
  \hline
  Factor categories & Estimate & Standard Error & t-value & p-value \\ 
  \hline
   High School or less - Bachelors or higher & -0.07 & 0.05 & -1.48 & 0.1395 \\ 
   Some college, diploma, assoc - Bachelors or higher & -0.05 & 0.05 & -1.08 & 0.2819 \\ 
   High School or less - Some college, diploma, assoc & -0.02 & 0.04 & -0.57 & 0.5690 \\ 
  \hline
\end{tabular}
\caption{Post-hoc test of education} 
\label{tab:DisabEducation}
\end{table}

% latex table generated in R 3.3.1 by xtable 1.8-2 package
% Wed Mar  8 13:33:07 2017
\begin{table}[H]
\footnotesize
\centering
\begin{tabular}{lrrrr}
  \hline
  Factor categories & Estimate & Standard Error & t-value & p-value \\ 
  \hline
    Male Married -  Female Not married & 0.29 & 0.06 & 4.92 & 0.0000 \\ 
    Female Married -  Female Not married & 0.31 & 0.05 & 6.69 & 0.0000 \\ 
    Male Not married -  Female Not married & 0.22 & 0.06 & 3.93 & 0.0001 \\ 
    Male Married -  Male Not married & 0.07 & 0.05 & 1.44 & 0.1509 \\ 
    Female Married -  Male Not married & 0.09 & 0.06 & 1.40 & 0.1629 \\ 
    Male Married -  Female Married & -0.02 & 0.06 & -0.35 & 0.7251 \\ 
  \hline
\end{tabular}
\caption{Post-hoc test of gender and marital status} 
\label{tab:DisabGenderMS}
\end{table}

% latex table generated in R 3.3.1 by xtable 1.8-2 package
% Wed Mar  8 13:35:12 2017
\begin{table}[H]
\footnotesize
\centering
\begin{tabular}{lrrrr}
  \hline
  Factor categories & Estimate & Standard Error & t-value & p-value \\ 
  \hline
    Married Black -  Not married Black & 0.37 & 0.07 & 5.04 & 0.0000 \\ 
    Married Black -  Not married White & 0.49 & 0.08 & 5.94 & 0.0000 \\ 
    Married White -  Not married White & 0.34 & 0.03 & 10.06 & 0.0000 \\ 
    Not married Hispanic -  Not married White & 0.31 & 0.08 & 3.74 & 0.0002 \\ 
    Married Black -  Not married Others & 0.40 & 0.12 & 3.34 & 0.0008 \\ 
    Married Hispanic -  Not married White & 0.25 & 0.08 & 3.16 & 0.0016 \\ 
    Not married Black -  Married White & -0.22 & 0.07 & -3.06 & 0.0022 \\ 
    Married Black -  Married Others & 0.30 & 0.12 & 2.58 & 0.0100 \\ 
    Not married Others -  Married White & -0.25 & 0.10 & -2.53 & 0.0113 \\ 
    Married Black -  Married Hispanic & 0.25 & 0.11 & 2.34 & 0.0190 \\ 
    Married Others -  Not married White & 0.20 & 0.10 & 2.05 & 0.0401 \\ 
    Not married Black -  Not married Hispanic & -0.19 & 0.10 & -1.88 & 0.0603 \\ 
    Married Black -  Married White & 0.15 & 0.08 & 1.85 & 0.0643 \\ 
    Not married Hispanic -  Not married Others & 0.22 & 0.12 & 1.84 & 0.0662 \\ 
    Not married Black -  Not married White & 0.12 & 0.07 & 1.72 & 0.0863 \\ 
    Married Black -  Not married Hispanic & 0.18 & 0.11 & 1.66 & 0.0975 \\ 
    Married Others -  Married White & -0.15 & 0.09 & -1.56 & 0.1186 \\ 
    Married Hispanic -  Not married Others & 0.16 & 0.12 & 1.32 & 0.1855 \\ 
    Not married Black -  Married Hispanic & -0.12 & 0.10 & -1.27 & 0.2028 \\ 
    Married Hispanic -  Married White & -0.09 & 0.08 & -1.23 & 0.2184 \\ 
    Married Others -  Not married Others & 0.11 & 0.10 & 1.00 & 0.3162 \\ 
    Not married Hispanic -  Married Others & 0.12 & 0.12 & 1.00 & 0.3182 \\ 
    Married Hispanic -  Not married Hispanic & -0.07 & 0.07 & -0.92 & 0.3560 \\ 
    Not married Others -  Not married White & 0.09 & 0.10 & 0.90 & 0.3678 \\ 
    Not married Black -  Married Others & -0.07 & 0.11 & -0.65 & 0.5154 \\ 
    Married Hispanic -  Married Others & 0.05 & 0.11 & 0.45 & 0.6497 \\ 
    Not married Hispanic -  Married White & -0.03 & 0.08 & -0.35 & 0.7267 \\ 
    Not married Black -  Not married Others & 0.03 & 0.11 & 0.30 & 0.7670 \\ 
  \hline
\end{tabular}
\caption{Post-hoc test of marital status with race/origin} 
\label{tab:DisabMSRaceOrigin}
\end{table}

% latex table generated in R 3.3.1 by xtable 1.8-2 package
% Wed Mar  8 13:37:15 2017
\begin{table}[H]
\footnotesize
\centering
\begin{tabular}{lrrrr}
  \hline
  Factor categories & Estimate & Standard Error & t-value & p-value \\ 
  \hline
    Female Black -  Female White & 0.24 & 0.07 & 3.17 & 0.0015 \\ 
    Male White -  Female White & 0.13 & 0.04 & 3.01 & 0.0026 \\ 
    Female Black -  Female Others & 0.32 & 0.12 & 2.76 & 0.0057 \\ 
    Male Hispanic -  Female White & 0.22 & 0.09 & 2.37 & 0.0180 \\ 
    Male Hispanic -  Female Others & 0.30 & 0.13 & 2.35 & 0.0188 \\ 
    Female Others -  Male White & -0.21 & 0.10 & -2.10 & 0.0359 \\ 
    Male Black -  Female Others & 0.25 & 0.13 & 1.98 & 0.0477 \\ 
    Male Others -  Female Others & 0.24 & 0.12 & 1.92 & 0.0544 \\ 
    Male Black -  Female White & 0.17 & 0.09 & 1.83 & 0.0680 \\ 
    Female Hispanic -  Female Others & 0.21 & 0.12 & 1.71 & 0.0866 \\ 
    Female Hispanic -  Female White & 0.13 & 0.08 & 1.52 & 0.1283 \\ 
    Male Others -  Female White & 0.16 & 0.11 & 1.48 & 0.1384 \\ 
    Female Black -  Male White & 0.11 & 0.08 & 1.42 & 0.1568 \\ 
    Female Black -  Female Hispanic & 0.11 & 0.10 & 1.07 & 0.2845 \\ 
    Male Hispanic -  Male White & 0.09 & 0.09 & 0.98 & 0.3286 \\ 
    Male Hispanic -  Female Hispanic & 0.09 & 0.10 & 0.94 & 0.3494 \\ 
    Female Others -  Female White & -0.08 & 0.10 & -0.82 & 0.4099 \\ 
    Male Black -  Female Black & -0.07 & 0.10 & -0.72 & 0.4742 \\ 
    Female Black -  Male Others & 0.08 & 0.12 & 0.67 & 0.5026 \\ 
    Male Hispanic -  Male Others & 0.06 & 0.13 & 0.49 & 0.6272 \\ 
    Male Black -  Male Hispanic & -0.05 & 0.12 & -0.43 & 0.6694 \\ 
    Male Black -  Male White & 0.04 & 0.09 & 0.41 & 0.6788 \\ 
    Male Black -  Female Hispanic & 0.04 & 0.12 & 0.35 & 0.7233 \\ 
    Male Others -  Male White & 0.03 & 0.11 & 0.25 & 0.8022 \\ 
    Female Hispanic -  Male Others & -0.03 & 0.13 & -0.23 & 0.8175 \\ 
    Female Black -  Male Hispanic & 0.02 & 0.11 & 0.15 & 0.8805 \\ 
    Male Black -  Male Others & 0.01 & 0.13 & 0.09 & 0.9274 \\ 
    Female Hispanic -  Male White & -0.00 & 0.08 & -0.03 & 0.9762 \\ 
   \hline
\end{tabular}
\caption{Post-hoc test of gender with race/origin} 
\label{tab:DisabGenderRaceOrigin}
\end{table}

\newpage
\bibliography{bibTex_Reference}
\end{document}
