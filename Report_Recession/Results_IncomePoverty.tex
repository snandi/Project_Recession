\documentclass[11pt]{extarticle} %extarticle for fontsizes other than 10, 11 And 12
%\documentclass[11p]{article}

%%%%%%%%%%%%%%%%%%%%%%%%%%%%%%%%%%%%%%%%%%%%%%%%%%%%%%%%%%%%%%%%%%%%%%
%% Input header file 
%%%%%%%%%%%%%%%%%%%%%%%%%%%%%%%%%%%%%%%%%%%%%%%%%%%%%%%%%%%%%%%%%%%%%%
%%%%%%%%%%%%%%%%%%%%%%%% Packages %%%%%%%%%%%%%%%%%%%%%%%%
\usepackage{amscd}
\usepackage{amsmath}
\usepackage{amssymb}
\usepackage{amsthm}
\usepackage{amsxtra}
\usepackage{animate}
\usepackage{bbold}
%\usepackage{bigints}
\usepackage{caption}    %% For multiple line captions
\usepackage{color, colortbl}
\usepackage{dsfont}
\usepackage{enumerate}
\usepackage[mathscr]{eucal}
%\usepackage{fancyhdr}
\usepackage{float}
%\usepackage{fullpage}  %% Dont use this for beamer presentations
\usepackage{geometry}
\usepackage{graphicx}
\usepackage{hyperref}
\usepackage{indentfirst}
\usepackage{latexsym}
\usepackage{listings}
\usepackage{longtable}  %% to add pagebreaks in between table
\usepackage{lscape}
\usepackage{mathtools}
\usepackage{microtype}
\usepackage{multirow}
\usepackage{natbib}
\usepackage{pdfpages}
\usepackage{setspace}   %% Allows to set double or single space
\usepackage{tcolorbox}  %% For colored textboxes
\usepackage{verbatim}
\usepackage{wrapfig}
\usepackage{xargs}
\usepackage{xcolor}
\DeclareGraphicsExtensions{.pdf,.png,.jpg, .jpeg}
\definecolor{LightCyan}{rgb}{0.88,1,1}

\usepackage{array}
\newcolumntype{C}[1]{>{\centering\arraybackslash}p{#1}}  %% For wrapping text in table headers

%%%%%%%%%%%%%%%%%%%%%%%% Commands %%%%%%%%%%%%%%%%%%%%%%%%
\newcommand{\Sup}{\textsuperscript}
\newcommand{\Exp}{\mathds{E}}
\newcommand{\Prob}{\mathds{P}}
\newcommand{\Z}{\mathds{Z}}
\newcommand{\Ind}{\mathds{1}}
\newcommand{\A}{\mathcal{A}}
\newcommand{\F}{\mathcal{F}}
%\newcommand{\G}{\mathcal{G}}
\newcommand{\I}{\mathcal{I}}
\newcommand{\R}{\mathcal{R}}
\newcommand{\Y}{\mathcal{Y}}
\newcommand{\Real}{\mathbb{R}}
\newcommand{\be}{\begin{equation}}
\newcommand{\ee}{\end{equation}}
\newcommand{\bes}{\begin{equation*}}
\newcommand{\ees}{\end{equation*}}
\newcommand{\union}{\bigcup}
\newcommand{\intersect}{\bigcap}
\newcommand{\Ybar}{\overline{Y}}
\newcommand{\ybar}{\bar{y}}
\newcommand{\Xbar}{\overline{X}}
\newcommand{\xbar}{\bar{x}}
\newcommand{\betahat}{\hat{\beta}}
\newcommand{\Yhat}{\widehat{Y}}
\newcommand{\yhat}{\hat{y}}
\newcommand{\Xhat}{\widehat{X}}
\newcommand{\xhat}{\hat{x}}
\newcommand{\E}[1]{\operatorname{E}\left[ #1 \right]}
%\newcommand{\Var}[1]{\operatorname{Var}\left( #1 \right)}
\newcommand{\Var}{\operatorname{Var}}
\newcommand{\Cov}[2]{\operatorname{Cov}\left( #1,#2 \right)}
\newcommand{\N}[2][1=\mu, 2=\sigma^2]{\operatorname{N}\left( #1,#2 \right)}
\newcommand{\bp}[1]{\left( #1 \right)}
\newcommand{\bsb}[1]{\left[ #1 \right]}
\newcommand{\bcb}[1]{\left\{ #1 \right\}}
\newcommand*{\permcomb}[4][0mu]{{{}^{#3}\mkern#1#2_{#4}}}
\newcommand*{\perm}[1][-3mu]{\permcomb[#1]{P}}
\newcommand*{\comb}[1][-1mu]{\permcomb[#1]{C}}
\newcommand{\indep}{\rotatebox[origin=c]{90}{$\models$}}

\DeclareMathOperator*{\argmin}{arg\,min}


\begin{document}
\doublespacing
%\SweaveOpts{concordance=TRUE}
\bibliographystyle{plain}  %Choose a bibliograhpic style

\title{Effects of Great Recession on Income Poverty}
\author{Subharati Ghosh, Subhrangshu Nandi, Susan Murphy \\
%  Statistics PhD Student, \\
%  Research Assistant,
%  Laboratory of Molecular and Computational Genomics, \\
%  University of Wisconsin - Madison}
%\date{February 16, 2015}
\date{}
}

\maketitle

\section{Study Aim}
The aim of this study is to analyze how households with a working age adult with disability compare with households with no adult with disability, during the {\emph{great recession}}, using ``Income Poverty'' as a measure of economic wellbeing, when controlling for demographic factors such as gender, marital status, education, race and origin. 

\section{Sample}
For this analysis data from US Census Bureau's SIPP 2008 panel survey was used. {\footnote{For more information on the SIPP 2008 panel schedule, please refer to this \href{http://www.census.gov/programs-surveys/sipp/data/2008-panel.html}{US Census Bureau website}}}. Questions on whether the households had a working age adult with disability were asked in wave-6 of the survey, which ended in August, 2010. Households that participated in wave-6 were included in our sample. There were a total of 34,850 households in wave six. Survey data upto wave-15 were used in our sample. Survey results from July, 2008 through June, 2013 were included in the analysis. Households whose reference person remained the same throughout the 2008 panel were kept in the sample. The reference persons of the households were also required to be 18 years or older throughout the 2008 panel. The final sample had 33,547 households that satisfied all the inclusion criteria. 

\section{Methods}
Total monthly household income was divided by monthly federal poverty level (FPL) and then averaged over quarters to estimate FPL100-ratio. An FPL100-ratio lower than one in any quarter indicated the household was below 100\% Federal poverty level in that quarter. Averaging monthly values over a quarter reduced the noise in the response variable by eliminating the month-by-month variability in the income data. In the sample, the monthly income data ranged from -\$27,180 to \$108,900, the average being \$5240 and median \$3,874. The negative incomes were associated with households owning business that incurred lossed in those months. The FPL100-ratio ranged from -17.95 to 89.48, with the average being 3.817 and the median 2.924. In the sample, 7,865 out of 33,547 households (23.44\%) had at least one working age adult with disability during the observed time period, as identified in wave-6 of the survey. Data from July 2008 (2008-Q3) through May 2013 (2013-Q1) were analyzed. This period overlapped with twelve of the eighteen months {\footnote{\href{http://www.nber.org/cycles/}{NBER Recession Cycles}}} of the ``Great recession'' and its long wake. 

\subsection{Mixed-effects model for income poverty}
A mixed (fixed and random) effects model was fit to estimate the impact of the presence of a working-age adult with disability in a household on its FPL100-ratio. Let $Y$ denote the vector or responses (FPL100-ratio). Let $\theta$ denote the vector of fixed effect factors like gender, marital status, education level, race, origin of household head, along with their interactions. Let $\beta$ denote another fixed effect of time, represented as quarters, starting from 2008-Q3 and ending in 2013-Q1. Let $b$ denote the household level random effect (random intercept). Then, the mixed-effects (\cite{Fitzmaurice_2012_Applied}) model for the responses, for each household $i$ can be written as
\begin{equation}
Y_{ij} = X_i\theta + \beta t_j + b_i + \epsilon_{ij}
\label{eq:MixedEffects1}
\end{equation}
where, $\epsilon_{ij}$ are regarded as measurement errors, $i$ goes from $1$ to $H$, the number of households, $j$ goes from $1$ to $T$, the total number of quarterly observations for every household. In this model, the response from the $i^{th}$ household at time $t_j$ is assumed to differ from the population mean $X_i\theta + \beta t_j$ by a household effect $b_i$ and a within household measurement error $\epsilon_{ij}$. The within household and between household errors are assumed to be normal and independent ($b_i \sim \N(0, \sigma^2_b),\ \ \epsilon_{ij} \sim \N(0, \sigma^2), \ \ b_i \indep \epsilon_{ij}, \forall i, j$). The effect of ``time'' is a fixed effect and it could be considered part of the fixed effect design matrix $X$. However, the problem of interest is to test a linear hypothesis about the disability-by-time interaction, to detect the effect of disability on the mean response over the period of the study. Hence, the ``time'' covariate is denoted separately. During estimation it will be estimated as a fixed effect. 

All analysis were conducted using the statistical software R (\cite{R}), version 3.3.1. The mixed effects models were fit using the R-package ``lme4'' (\cite{R-lme4}) and all hypothesis tests were done using the R package ``lmerTest'' (\cite{Kuznetsova_etal_2015_R-lmerTest}). The final model was fit with some of the fixed effect factors along with their interactions after performing ``backward elimination'' on the full model. Elimination of the fixed effects were done by the principle of marginality, that is: the highest order interactions are tested first: if they are significant, the lower order effects were included in the model without testing for significance. The p-values for the fixed effects are estimated from the F statistics, with ``Satterthwaite'' approximation (\cite{Satterthwaite_1946_Biometrics}) denominator degrees of freedom. The p-values for the random effect were computed from likelihood ratio tests (\cite{Morrell_1998_Biometrics}). 

\subsection{Model for income poverty, with baseline correction}
In order to isolate the effect of the {\emph{great recession}} on income poverty (response), measured by FPL100-ratio, the values of the response at the beginning of study period (2008-Q3) were subtracted from each household's responses. Consequently, all FPL100-ratios of all households at 2008-Q3 were zero. The same model as in eq \ref{eq:MixedEffects1} was fit to this baseline-corrected responses. The problem of interest was to test a linear hypothesis about the disability-by-time interaction. 

\section{Results}
ANOVA TABLE of first model\\

% latex table generated in R 3.3.1 by xtable 1.8-2 package
% Tue Jan 31 19:39:05 2017
\begin{table}[ht]
\footnotesize
\centering
\begin{tabular}{lrrrrrr}
  \hline
 & Sum Sq & Mean Sq & NumDF & DenDF & F.value & Pr($>$F) \\ 
  \hline
  yearqtrNum & 145.47 & 145.47 & 1.00 & 541289.64 & 12.07 & 0.0005 \\ 
  race\_origin & 5752.68 & 2876.34 & 3.00 & 61173.45 & 238.71 & 0.0000 \\ 
  adult\_disb & 4091.86 & 4091.86 & 1.00 &  &  &  \\ 
  education & 29404.45 & 14702.23 & 2.00 & 138172.11 & 1220.15 & 0.0000 \\ 
  gender:race\_origin & 104.06 & 34.69 & 3.00 & 171428.50 & 2.88 & 0.0345 \\ 
  gender:ms & 1813.16 & 1813.16 & 1.00 & 357150.73 & 150.47 & 0.0000 \\ 
  ms:race\_origin & 2070.73 & 690.24 & 3.00 & 390659.60 & 57.28 & 0.0000 \\ 
  race\_origin:adult\_disb & 26.90 & 8.97 & 3.00 & 61969.87 & 0.74 & 0.5255 \\ 
  gender:adult\_disb & 237.05 & 237.05 & 1.00 & 155731.47 & 19.67 & 0.0000 \\ 
  ms:adult\_disb & 0.12 & 0.12 & 1.00 & 373284.89 & 0.01 & 0.9222 \\ 
  yearqtrNum:adult\_disb & 145.47 & 145.47 & 1.00 & 541289.64 & 12.07 & 0.0005 \\ 
  adult\_disb:education & 109.34 & 54.67 & 2.00 & 138155.91 & 4.54 & 0.0107 \\ 
  gender:ms:adult\_disb & 56.78 & 56.78 & 1.00 & 354664.50 & 4.71 & 0.0299 \\ 
   \hline
\end{tabular}
\end{table}

ANOVA TABLE of model with no baseline\\
% latex table generated in R 3.3.1 by xtable 1.8-2 package
% Tue Jan 31 19:46:11 2017
\begin{table}[ht]
\footnotesize
\centering
\begin{tabular}{lrrrrrr}
  \hline
 & Sum Sq & Mean Sq & NumDF & DenDF & F.value & Pr($>$F) \\ 
  \hline
  yearqtrNum & 138.47 & 138.47 & 1.00 & 544073.07 & 11.52 & 0.0007 \\ 
  race\_origin & 151.47 & 75.73 & 3.00 & 55510.96 & 6.30 & 0.0018 \\ 
  adult\_disb & 22.10 & 22.10 & 1.00 &  &  &  \\ 
  education & 72.76 & 36.38 & 2.00 & 107788.31 & 3.03 & 0.0485 \\ 
  gender:race\_origin & 181.68 & 60.56 & 3.00 & 131972.69 & 5.04 & 0.0017 \\ 
  gender:ms & 721.70 & 721.70 & 1.00 & 278695.88 & 60.04 & 0.0000 \\ 
  ms:race\_origin & 556.33 & 185.44 & 3.00 & 312406.87 & 15.43 & 0.0000 \\ 
  race\_origin:adult\_disb & 21.56 & 7.19 & 3.00 & 55461.90 & 0.60 & 0.6163 \\ 
  gender:adult\_disb & 19.75 & 19.75 & 1.00 & 120125.79 & 1.64 & 0.1999 \\ 
  ms:adult\_disb & 31.83 & 31.83 & 1.00 & 292894.97 & 2.65 & 0.1037 \\ 
  yearqtrNum:adult\_disb & 138.47 & 138.47 & 1.00 & 544073.07 & 11.52 & 0.0007 \\ 
  adult\_disb:education & 35.49 & 17.75 & 2.00 & 107764.38 & 1.48 & 0.2285 \\ 
  gender:ms:adult\_disb & 14.38 & 14.38 & 1.00 & 276123.49 & 1.20 & 0.2740 \\ 
   \hline
\end{tabular}
\end{table}

% Table created by stargazer v.5.2 by Marek Hlavac, Harvard University. E-mail: hlavac at fas.harvard.edu
% Date and time: Tue, Jan 31, 2017 - 07:28:23 PM
\begin{table}[!htbp] \centering 
\tiny
  \caption{} 
  \label{} 
\begin{tabular}{@{\extracolsep{5pt}}lcc} 
\\[-1.8ex]\hline 
\hline \\[-1.8ex] 
 & \multicolumn{2}{c}{\textit{Dependent variable:}} \\ 
\cline{2-3} 
\\[-1.8ex] & FPL100\_num & FPL100\_noBaseline \\ 
\\[-1.8ex] & (1) & (2)\\ 
\hline \\[-1.8ex] 
 yearqtrNum & $-$0.05$^{***}$ & $-$0.05$^{***}$ \\ 
  & (0.002) & (0.002) \\ 
  & & \\ 
 genderFemale & $-$0.13$^{**}$ & 0.16$^{**}$ \\ 
  & (0.07) & (0.06) \\ 
  & & \\ 
 msNot married & $-$0.24$^{***}$ & $-$0.11$^{**}$ \\ 
  & (0.05) & (0.05) \\ 
  & & \\ 
 race\_originHispanic & $-$0.11 & $-$0.10 \\ 
  & (0.09) & (0.08) \\ 
  & & \\ 
 race\_originOthers & 0.67$^{***}$ & $-$0.14 \\ 
  & (0.10) & (0.09) \\ 
  & & \\ 
 race\_originWhite & 1.10$^{***}$ & $-$0.01 \\ 
  & (0.07) & (0.07) \\ 
  & & \\ 
 adult\_disbyes & $-$30.84$^{***}$ & $-$29.69$^{***}$ \\ 
  & (8.74) & (8.72) \\ 
  & & \\ 
 educationSome college, diploma, assoc & 0.52$^{***}$ & $-$0.06$^{**}$ \\ 
  & (0.03) & (0.03) \\ 
  & & \\ 
 educationBachelors or higher & 1.87$^{***}$ & 0.04 \\ 
  & (0.03) & (0.03) \\ 
  & & \\ 
 genderFemale:race\_originHispanic & $-$0.05 & $-$0.17$^{**}$ \\ 
  & (0.09) & (0.08) \\ 
  & & \\ 
 genderFemale:race\_originOthers & $-$0.31$^{***}$ & $-$0.38$^{***}$ \\ 
  & (0.11) & (0.10) \\ 
  & & \\ 
 genderFemale:race\_originWhite & $-$0.10 & $-$0.17$^{***}$ \\ 
  & (0.07) & (0.06) \\ 
  & & \\ 
 genderFemale:msNot married & $-$0.55$^{***}$ & $-$0.32$^{***}$ \\ 
  & (0.04) & (0.04) \\ 
  & & \\ 
 msNot married:race\_originHispanic & 0.35$^{***}$ & 0.15$^{**}$ \\ 
  & (0.06) & (0.06) \\ 
  & & \\ 
 msNot married:race\_originOthers & $-$0.05 & 0.06 \\ 
  & (0.08) & (0.08) \\ 
  & & \\ 
 msNot married:race\_originWhite & $-$0.24$^{***}$ & $-$0.13$^{***}$ \\ 
  & (0.05) & (0.05) \\ 
  & & \\ 
 race\_originHispanic:adult\_disbyes & 0.06 & 0.10 \\ 
  & (0.13) & (0.12) \\ 
  & & \\ 
 race\_originOthers:adult\_disbyes & 0.08 & 0.15 \\ 
  & (0.15) & (0.13) \\ 
  & & \\ 
 race\_originWhite:adult\_disbyes & $-$0.05 & 0.05 \\ 
  & (0.10) & (0.09) \\ 
  & & \\ 
 genderFemale:adult\_disbyes & 0.14$^{**}$ & 0.02 \\ 
  & (0.06) & (0.06) \\ 
  & & \\ 
 msNot married:adult\_disbyes & $-$0.09 & 0.02 \\ 
  & (0.06) & (0.05) \\ 
  & & \\ 
 yearqtrNum:adult\_disbyes & 0.02$^{***}$ & 0.01$^{***}$ \\ 
  & (0.004) & (0.004) \\ 
  & & \\ 
 adult\_disbyes:educationSome college, diploma, assoc & $-$0.10$^{*}$ & 0.09$^{*}$ \\ 
  & (0.06) & (0.05) \\ 
  & & \\ 
 adult\_disbyes:educationBachelors or higher & $-$0.21$^{***}$ & 0.04 \\ 
  & (0.07) & (0.07) \\ 
  & & \\ 
 genderFemale:msNot married:adult\_disbyes & 0.17$^{**}$ & 0.08 \\ 
  & (0.08) & (0.07) \\ 
  & & \\ 
 (Intercept) & 106.32$^{***}$ & 106.18$^{***}$ \\ 
  & (4.15) & (4.14) \\ 
\hline\\[-1.8ex]
{\bf Random Effects} & Std.Dev. & Std.Dev.\\
\hline\\[-1.8ex]
hhid - (Intercept) & 2.80 & 2.36\\
\\
Residual Standard Deviation & 3.47 & 3.47\\
  & & \\ 
\hline \\[-1.8ex] 
Observations & 570,006 & 570,006 \\ 
Log Likelihood & $-$1,247,068.00 & $-$1,240,888.00 \\ 
Akaike Inf. Crit. & 2,494,192.00 & 2,481,833.00 \\ 
Bayesian Inf. Crit. & 2,494,507.00 & 2,482,148.00 \\ 
\hline 
\hline \\[-1.8ex] 
\textit{Note:}  & \multicolumn{2}{r}{$^{*}$p$<$0.1; $^{**}$p$<$0.05; $^{***}$p$<$0.01} \\ 
\end{tabular} 
\end{table} 


\section{Limitations}
\begin{enumerate}
\item Although a linear mixed effects regression model discovered some conventional and some interesting patterns in the relationships between response and demographic factors, along with disability, the trajectory of income poverty over the study period for some households were not linear. This modeling approach does not capture trajectory shapes of individual households. A non-parametric fitting of the income poverty trajectories could be tried as a pre-processing step before testing for differences in behavior between different groups of households. 
\end{enumerate}

\newpage
\bibliography{bibTex_Reference}
\end{document}
